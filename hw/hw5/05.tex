\documentclass[10pt, twocolumn]{article}
\author{Tarang Srivastava}
\usepackage{amsmath, amssymb, amsthm, commath, chngcntr, enumitem, multirow, thmtools, xcolor}
\usepackage{graphicx}
\usepackage[margin=.25in]{geometry}
\setlength{\columnsep}{.5in}
\newcommand{\C}{\mathbb{C}}
\newcommand{\question}[1]{\textcolor{blue}{#1} \\}
\newcommand{\R}{\mathbb{R}}
\newcommand{\F}{\mathbb{F}}
\newcommand{\N}{\mathbb{N}}
\newcommand{\LinearMap}[2]{\mathcal{L}(#1, #2)}
\newcommand{\poly}[2]{\mathcal{P}_{#1}\left(#2\right)}
\newcommand{\vspan}[1]{\text{span}\left(#1\right)}
\newcommand{\inv}[1]{#1^{-1}}
\newcommand{\todo}[1]{\textcolor{red}{TODO: #1} \\}
\newcommand{\nul}{\text{null }}
\newcommand{\nullity}{\text{nullity }}
\newcommand{\range}{\text{range }}
\newcommand{\rank}{\text{rank }}
\newcommand{\makechaptertitle}[1]{
\begin{center}
	\begin{large}
		#1
	\end{large}
	\begin{small}
		\\Tarang Srivastava
	\end{small}
\end{center}
}
\declaretheoremstyle[
spaceabove=\topsep, spacebelow=\topsep,
headfont=\normalfont\bfseries,
notefont=\bfseries, notebraces={Problem }{},
bodyfont=\normalfont,
postheadspace=0.5em,
name={\ignorespaces},
numbered=no,
headpunct=:]
{mystyle}
\declaretheorem[style=mystyle]{q}

\begin{document}
	
\makechaptertitle{Math 110 Homework 5}

\section{Exercises 3.E}
\begin{q}[3]
    
\end{q}
\begin{q}[10]
    From last discussion we know that for any affine subset $ A $ and vectors $ u, v \in A $. 
    It is true that $ \lambda u + (1 - \lambda) v \in A $. 
    Then, it follows directly that for affines subsets $ A_1 $ and $ A_2 $ if the intersection is not empty, 
    $ u, v \in A_1 \cap A_2 $, then $ \lambda u + (1 - \lambda) v \in A_1 $. 
    And, similarily $ \lambda u + (1 - \lambda) v \in A_2 $.  
    Therefore, $ A_1 $ and $ A_2 $ are affine subsets and we can define their intersection with the following values to also be a affine subset with the following form. 
    If the intersection is empty, then clearly we get that the following form would be empty as well, therefore the intersection of the affine subsets would be empty.
\end{q}
\begin{q}[11]

    
\end{q}

\end{document}