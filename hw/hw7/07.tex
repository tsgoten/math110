\documentclass[10pt, twocolumn]{article}
\author{Tarang Srivastava}
\usepackage{amsmath, amssymb, amsthm, commath, chngcntr, enumerate, multirow, thmtools, xcolor}
\usepackage{graphicx}
\usepackage[margin=.25in]{geometry}
\setlength{\columnsep}{.5in}
\newcommand{\C}{\mathbb{C}}
\newcommand{\question}[1]{\textcolor{blue}{#1} \\}
\newcommand{\R}{\mathbb{R}}
\newcommand{\F}{\mathbb{F}}
\newcommand{\N}{\mathbb{N}}
\newcommand{\LinearMap}[2]{\mathcal{L}(#1, #2)}
\newcommand{\poly}[2]{\mathcal{P}_{#1}\left(#2\right)}
\newcommand{\vspan}[1]{\text{span}\left(#1\right)}
\newcommand{\inv}[1]{#1^{-1}}
\newcommand{\todo}[1]{\textcolor{red}{TODO: #1} \\}
\newcommand{\nul}{\text{null }}
\newcommand{\nullity}{\text{nullity }}
\newcommand{\range}{\text{range }}
\newcommand{\rank}{\text{rank }}
\newcommand{\annhilator}[1]{#1^{0}}
\newcommand{\makechaptertitle}[1]{
\begin{center}
	\begin{large}
		#1
	\end{large}
	\begin{small}
		\\Tarang Srivastava
	\end{small}
\end{center}
}
\declaretheoremstyle[
spaceabove=\topsep, spacebelow=\topsep,
headfont=\normalfont\bfseries,
notefont=\bfseries, notebraces={Problem }{},
bodyfont=\normalfont,
postheadspace=0.5em,
name={\ignorespaces},
numbered=no,
headpunct=:]
{mystyle}
\declaretheorem[style=mystyle]{q}

\begin{document}
\makechaptertitle{Math 110 Homework 6}

\section{Exercises 5.A}
\begin{q}[1]
	The argument is as follows.
	\begin{enumerate}[(a)]
		\item Let $ u $ be an arbitrary vector $ u \in U $.
		If $ U \subset \nul T $, then $ u \in \nul T $. 
		So, $ Tu = 0 $. Since, $ U $ is a vector space, it must be that $ 0 \in U $, so $ Tu \in U $.
		Thus, $ U $ is invariant under $ T $ given the condition.
		\item By definition we have $ Tu \in \range T $ for all $ u \in U $. 
		Since, $ \range T \subset U $ we have that for all $ u $, $ Tu \in U $. 
		Thus, $ U $ is invariant under $ T $ given the condition.
	\end{enumerate}
\end{q}



\end{document}