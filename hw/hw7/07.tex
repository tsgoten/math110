\documentclass[10pt, twocolumn]{article}
\author{Tarang Srivastava}
\usepackage{amsmath, amssymb, amsthm, commath, chngcntr, enumerate, multirow, thmtools, xcolor}
\usepackage{graphicx}
\usepackage[margin=.25in]{geometry}
\setlength{\columnsep}{.5in}
\newcommand{\C}{\mathbb{C}}
\newcommand{\question}[1]{\textcolor{blue}{#1} \\}
\newcommand{\R}{\mathbb{R}}
\newcommand{\F}{\mathbb{F}}
\newcommand{\N}{\mathbb{N}}
\newcommand{\LinearMap}[2]{\mathcal{L}(#1, #2)}
\newcommand{\Operator}[1]{\mathcal{L}(#1)}
\newcommand{\poly}[2]{\mathcal{P}_{#1}\left(#2\right)}
\newcommand{\vspan}[1]{\text{span}\left(#1\right)}
\newcommand{\inv}[1]{#1^{-1}}
\newcommand{\todo}[1]{\textcolor{red}{TODO: #1} \\}
\newcommand{\nul}{\text{null }}
\newcommand{\nullity}{\text{nullity }}
\newcommand{\range}{\text{range }}
\newcommand{\rank}{\text{rank }}
\newcommand{\annhilator}[1]{#1^{0}}
\newcommand{\makechaptertitle}[1]{
\begin{center}
	\begin{large}
		#1
	\end{large}
	\begin{small}
		\\Tarang Srivastava
	\end{small}
\end{center}
}
\declaretheoremstyle[
spaceabove=\topsep, spacebelow=\topsep,
headfont=\normalfont\bfseries,
notefont=\bfseries, notebraces={Problem }{},
bodyfont=\normalfont,
postheadspace=0.5em,
name={\ignorespaces},
numbered=no,
headpunct=:]
{mystyle}
\declaretheorem[style=mystyle]{q}

\begin{document}
\makechaptertitle{Math 110 Homework 6}

\section{Exercises 5.A}
\begin{q}[1]
	The argument is as follows.
	\begin{enumerate}[(a)]
		\item Let $ u $ be an arbitrary vector $ u \in U $.
		If $ U \subset \nul T $, then $ u \in \nul T $. 
		So, $ Tu = 0 $. Since, $ U $ is a vector space, it must be that $ 0 \in U $, so $ Tu \in U $.
		Thus, $ U $ is invariant under $ T $ given the condition.
		\item By definition we have $ Tu \in \range T $ for all $ u \in U $. 
		Since, $ \range T \subset U $ we have that for all $ u $, $ Tu \in U $. 
		Thus, $ U $ is invariant under $ T $ given the condition.
	\end{enumerate}
\end{q}
\begin{q}[3]
	We wish to show that for all $ u \in \range S $ we have that $ Tu \in \range S $. 
	Let $ v \in V $, then $ STv \in \range S $ by definition. 
	Given $ ST = TS $, we have that $ STv = TSv $. 
	So, $ TSv \in \range S $. 
	Let $ u \in \range S $, then there exists some $ v \in V $ such that $ Sv = u $. 
	Since,$ TSv \in \range S $, we have $ Tu \in \range S $.  
\end{q}
\begin{q}[6]
	True! \\
	We have a subspace $ U $ of $ V $ such that it is invariant for all $ T \in \LinearMap{V}{V} $, 
	assume for contradiction that $ U \neq 0 $ and $ U \neq V $. 
	Then, since $ V $ is finite dimensional we have some basis of $ U $ 
	$$ u_1, ..., u_m \text{ is a basis of } U $$
	Then we can extend the basis of $ U $ to a basis of $ V $, and since we know that $ U \neq V $ it must be that we must extend it by at least one vector.
	$$ u_1, ..., u_m, v_1, ..., v_n \text{ is a basis of } V $$
	Then, let $ T \in \Operator{V} $ such that for all $ i \in 1, ..., n $ we have that 
	$$ Tu_i = v_i $$
	and the remaining basis vectors of $ U $, if there are any, are mapped to 0.
	Let $ u $ be an arbitrary vector $ u \in U $, then 
	\begin{align*}
		u &= a_1 u_1 + ... + a_m u_m 
		\intertext{for some scalars $ a_1, ..., a_m $. Then, }	
		Tu &= a_1Tu_1 + ... + a_m Tu_m \\
		Tu &= a_1v_1 + ... + a_m v_m 
		\intertext{Since we have that $ U $ is invariant under all linear maps it must be that $ Tu \in U $ so there exists some linear combination of $ u_1, ..., u_m $ that is equal to $ Tu $. So for some scalars $ b_1, .., b_m $}
		Tu &= b_1u_1 + ... + b_mu_m 
		\intertext{Substituting the two reprsentations of $ Tu $ we get}
		b_1u_1 + ... + b_mu_m &= a_1v_1 + ... + a_m v_m
		\intertext{Then, we have a contradiction since we claimed that $ u_1, ..., u_m, v_1, ..., v_n $ is a basis and therefore a linearly independent list of vectors. But since they can be expressed as a linear combination of each other as such they are not linearly independent, by some previous exercises. Thus, it must be that $ U = \{0\} $ or $ U = V $.}
	\end{align*}
\end{q}
\begin{q}[8]
	By defintion we wish to find eigenvalues and eigenvectors, $ v = (w, z) $ such that 
	$$ T (w, z) = (z, w) = \lambda (w, z) = (\lambda w, \lambda z) $$
	Then, we have to find solutions to $ \lambda w = z $ and $ \lambda z = w $. 
	Following some substitutions we get 
	$$ z(\lambda^2 - 1) = 0 $$
	Since, $ v \neq 0 $ we are left with $ \lambda = \pm 1 $. \\
	So, $ \lambda_1 = 1 $ with the corresponding eigenvectors some scalar multiple $ v_1 = (1, 1)$
	and $ \lambda_2 = -1 $ with the corresponding eigenvectors some scalar multiple $ v_2 = (-1, 1) $.
\end{q}
\begin{q}[12]
	We wish to find eigenvalues and eigenvectors, 
	$ p(x) = a_4x^4 + a_3x^3 + a_2x^2 + a_1x + a_0 $, such that
	$$ (Tp)(x) = x p'(x) = \lambda p(x) $$
	That is, 
	$ 4 a_4x^4 +  3a_3x^3 + 2 a_2x^2 + a_1x = \lambda a_4x^4 +\lambda a_3x^3 +\lambda a_2x^2 +\lambda a_1x +\lambda a_0 $.
	So, clearly $ 4 a_4 x^4 = \lambda a_4 x^4 $. Solving for this we get $ \lambda = 4 $, but then the following terms do not hold so we must have that $ a_3 = a_2 = a_1 = a_0 = 0 $. 
	So the polynomial for $ \lambda = 4 $ must be of the form $a_4 x^4 $. 
	We follow with this argument for the remaining terms to get that the eigenvalues are 
	$ \lambda = 4, 3, 2, 1 $ and that the corresponding eigenvectors are some scalar multiple of $ x^4, x^3, x^2, x $ respectively.
\end{q}
\begin{q}[13]
	We can just show that $ \alpha - \lambda \leq |\alpha - \lambda| < \frac{1}{1000} $ is equivalent to showing 
	$ \alpha < \frac{1}{1000} + \lambda $ since we are working with elements in our field and performing field operation it must be the case that $ \alpha \in \F $. (Clearly this wont hold in ALL fields but in the world of Axler this is just $ \C $ or $ \R $).
	Then since $ V $ is finite dimension there are at most $ \dim V $ many eigenvalues, that is finite number of eigenvalues. 
	Again, since we are either working with either the reals or complex, which are both infinite just select a number $ \alpha $ that satisfies the inequality and is not eqaul to one of the eigenvalues. 
	Then, it follows by 5.6, since we chose $ \alpha $ such that it is not an eigenvalue $ T - \alpha I $ is invertible.
\end{q}
\begin{q}[15]
	(a)
	Let $ \lambda $ be an eigenvalue of $ T $. Then, $ Tv = \lambda v $ for some corresponding eigenvector $ v $. 
	Let, $ u $ be a vector in $ V $ such that $ Su = v $. We know this exists, since $ S $ is an invertible operator on $ V $. 
	Therefore,
	\begin{align*}
		Tv =& \lambda v \\
		TSu &= \lambda Su 
		\intertext{Then composing with $ \inv{S} $ we get}
		\inv{S}TSu &= \inv{S} \lambda S u \\
		&= \lambda \inv{S} S u \\
		&= \lambda u
	\end{align*}
	Thus, the eigenvalues are the same. \\
	(b) From our argumetn above we had that the corresponding eigenvectors $ v $ of $ T $ have 
	the eigenvector $ u $ for $ \inv{S} T S $ such that $ v = Su $ or $ \inv{S} v = u $. That is the relationship.
\end{q}
\begin{q}[19]
	We have that for $ T(1, 1, ..., 1) = (n, n , ..., n) = \lambda (1, 1, ..., 1) $ if we have $ \lambda = n $ we satisfy the inequality.
	So, the corresponding eigenvector to $ \lambda = n $ is any scalar multiple of $ (1, ..., 1) $. 
	These are all the eigenvalues and eigenvectors of $ T $, since by Kubrat's hint the Trace is equal to the sum of the eigen values and the trace is eqaul to $ n $ if there are only 1s in the main diagonal. 
	Since, we found an eigenvalue equal to $ n $ thre can be no more, so we are done. 
\end{q}




\end{document}