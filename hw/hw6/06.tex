\documentclass[10pt, twocolumn]{article}
\author{Tarang Srivastava}
\usepackage{amsmath, amssymb, amsthm, commath, chngcntr, enumitem, multirow, thmtools, xcolor}
\usepackage{graphicx}
\usepackage[margin=.25in]{geometry}
\setlength{\columnsep}{.5in}
\newcommand{\C}{\mathbb{C}}
\newcommand{\question}[1]{\textcolor{blue}{#1} \\}
\newcommand{\R}{\mathbb{R}}
\newcommand{\F}{\mathbb{F}}
\newcommand{\N}{\mathbb{N}}
\newcommand{\LinearMap}[2]{\mathcal{L}(#1, #2)}
\newcommand{\poly}[2]{\mathcal{P}_{#1}\left(#2\right)}
\newcommand{\vspan}[1]{\text{span}\left(#1\right)}
\newcommand{\inv}[1]{#1^{-1}}
\newcommand{\todo}[1]{\textcolor{red}{TODO: #1} \\}
\newcommand{\nul}{\text{null }}
\newcommand{\nullity}{\text{nullity }}
\newcommand{\range}{\text{range }}
\newcommand{\rank}{\text{rank }}
\newcommand{\annhilator}[1]{#1^{0}}
\newcommand{\makechaptertitle}[1]{
\begin{center}
	\begin{large}
		#1
	\end{large}
	\begin{small}
		\\Tarang Srivastava
	\end{small}
\end{center}
}
\declaretheoremstyle[
spaceabove=\topsep, spacebelow=\topsep,
headfont=\normalfont\bfseries,
notefont=\bfseries, notebraces={Problem }{},
bodyfont=\normalfont,
postheadspace=0.5em,
name={\ignorespaces},
numbered=no,
headpunct=:]
{mystyle}
\declaretheorem[style=mystyle]{q}

\begin{document}
	
\makechaptertitle{Math 110 Homework 6}

\section{Exercises 3.F}
\begin{q}[7]
    Since, $ p $ is in the standard basis of $ \poly{m}{\R} $ it has only one term to consider.
    Namely, a term $ x^j $ where $ j \in \{1, ..., m\} $. 
    Then, if $ p $ has a degree strictly greater than $ j $ it will have some $ x $ term for $ p^(j) $. 
    Thus, evaluated at 0, it will equal 0. 
    If $ p $ has a degree strictly less than $ j $ then $ p^{(j)} = 0 $ by the power rule form calculus.
    So, when $ p $ has degree exactly $ j $ we have that it will evaluate to some constant by the power rule. 
    This constant will be exactly $ j! $, again by the power rule. 
    So, we normalize it by dividing it by $ j! $, so $ \varphi(j) = 1$. 
    Then, we have a valid $ \varphi_j $ such that it only evaluates to 1 for one of the polynomials in the standard basis and 0 for all other. 
    Thus, a valid dual basis. 
\end{q}
\begin{q}[9]
    Since, $ \psi \in V' $ it is equivalent to the following linear combination with scalars $ a_1, .., a_n $. 
    That is, 
    $$ \psi = a_1 \varphi_1 + ... + a_n \varphi_n $$
    Then for some $ v_j $ in the basis of $ V $ w ehave that 
    $$ \psi(v_j) = a_j \varphi_j(v_j) = a_j $$
    Since, all the other ones are 0, since it is evaluated at a basis vector
    Then, we simply make direct substitutions for each $ j \in \{1, ..., n\} $ and get the desired result.
\end{q}
\begin{q}[11]
    This is pretty simple. 
    For the $ \impliedby $ direction. 
    All $ d_j $ are some arbitrary costants in $ \F $. 
    So, we construct $ A_{j, k} $ such that each column is just a scalar multiple of each other. 
    Since, we are multiplying the same $ (c_1, ..., c_m $ by arbitrary constants $ d_j $. 
    Then, clearly if all the vectors are scalar multiples of each other we must remove all of them except 1 to get a linearly independent list. 
    Therefore, the dimension of the row space is 1 and thus rank is 1. 
    For the $ \implies $ direction we argue that since $ A $ has rank 1, then the column space has dimesnion 1. 
    So, it follows that the columns are constructed as follows for some vectors $ d_1, ..., d_n $. That is they are scalar multiples of each other.
\end{q}
\begin{q}[19]
    $ \implies $. If $ U = V $, then the $ \varphi $ such that $ \varphi(u) = 0 = \varphi(v) $ for all $ u \in U$ and all $ v \in V $ is just 0, since 0 is unique in a vector space. \\
    $ \impliedby $. We know $ V $ is finite-dimensional, so 
    $$ \dim U + \dim \annhilator{U}  = \dim V $$
    But, the annhilator has dimension 0 so 
    $ \dim U = \dim V $
    since,  $ U $ is a subspace of $ V $ it implies that 
    $ U = V $
\end{q}
\begin{q}[20]
    Pick an arbitrary $ \varphi \in \annhilator{W} $. 
    Since, $ U \subset W $ for all $ u \in U $, $ \varphi(u) = 0 $, so clearly $ \varphi \in \annhilator{U} $. 
    We showed this for an arbitrary $ \varphi $ so $ \annhilator{W} \subset \annhilator{U} $.
\end{q}
\begin{q}[21]
    Since $ V $ is finite dimensional we abuse the dimension of the annhilator formula.  
    \begin{align*}
        \dim U + \dim \annhilator{U}  &= \dim V \\
        \dim W + \dim \annhilator{W}  &= \dim V \\
        \dim W + \dim \annhilator{W}  &= \dim U + \dim \annhilator{U} \\
        \intertext{Since, $ \annhilator{W} \subset \annhilator{U} $ we have that}
        \dim \annhilator{W} &\leq \dim \annhilator{U} \\
        \intertext{Then, to hold the equality it must be the case that}
        \dim W &\geq \dim U
        \intertext{Therefore,}
        W &\supset U
    \end{align*}
\end{q}
\begin{q}[22]
    For some $ v \in V + U $ we have that $ v = u + w $ for $ u \in U $ and $ w \in W $. 
    Then for some $ \varphi \in (U + W)^{0} $
    , $ \varphi (v) = 0  = \varphi(u + w) = \varphi(u) + \varphi(w) = 0 $.
    Since, $ u, w \in U + W $ then clearly they must be zero as well. 
    So, $ \varphi \in \annhilator{U} \cap \annhilator{W} $. 
    Thus, $ \annhilator{U + W} \subset \annhilator{U} \cap \annhilator{W} $. 
    We use the exact same argument in the reverse direction to get the other conclusion 
    $ \annhilator{U + W} \supset \annhilator{U} \cap \annhilator{W} $
    So, then it must be that
    $ \annhilator{U + W} = \annhilator{U} \cap \annhilator{W} $
\end{q}
\begin{q}[23]
    Since, $ V $ is finite dimensional and $ U $ and $ W $ are subspaces we know 
    $$ \dim V = \dim U + \dim W - \dim U \cap W $$
    by adding extra $ \dim V $ on each sides and moving things around we get
    $$ \dim V - \dim U + \dim V - \dim W = \dim V - \dim U \cap W $$
    We also know that $ \dim \annhilator{U} + \dim U = V $ and $ \dim \annhilator{W} + \dim W = V $ and 
    $ \dim (U \cap W) + \dim \annhilator{(U \cap W)} = \dim V $. 
    Then making the right substitutions we get, 
    $$ \dim \annhilator{U} + \dim \annhilator{W} = \dim \annhilator{(U \cap W)} $$
    To make it all work, we just have to show that one of the sides is a subspace of the other. 
    Suppose we have a $ \varphi \in (U \cap W)^0 $, then for some $ \psi in U^0 + W^0 $ we know that $ \psi = \varphi_U + \varphi_W $ for $ \varphi_U \in \annhilator{U}  $ and $ \varphi_W \in \annhilator{W} $. 
    Then, clealry $ \psi(v) $ for some $ v\in U \cap W $ it holds that $ \psi(v) = (\varphi_U + \varphi_W)(v) = \varphi_U(v) + \varphi_W(v) $ and since $ v $ is in $ U $ and $ W $ it must be that $ \varphi_U(v) = 0 $ and $ \varphi_W(v) = 0 $ so $ \psi(v) = 0 $. 
    So, we showed that 
    $$ (U \cap W)^0  \subset U^0 + W^0 $$
    Therefore, 
    $$ (U \cap W)^0  = U^0 + W^0 $$
\end{q}
\begin{q}[34]
    \begin{enumerate}
        \item It is trivial to show that some $ phi $ plus another one will result in one that is defined if we just did it as before. The same for the scalar multiplication. 
        \item Since we follow our previous defintion of dual basis, that is $ V'' = \LinearMap{V'}{\R} $, then it must be by the same argument $ \dim V = \dim V' = \dim V $. Since, t
        \item some argument.
    \end{enumerate} 
\end{q}
\begin{q}[35]
    We define a linear map $ T : \poly{}{\R}' \to \R^\infty $ as follows. 
    By our understanding of the standard dual basis of $ \poly{}{\R}' $ for some $ \varphi \in \poly{}{\R}' $ is of the form $ \varphi = a_1 \varphi_1 + a_2 \varphi_2 + ... $. 
    So we define $ T(\varphi) = (a_1, a_2, ....) $. We know the following coefficients uniquely, so the mapping is to a unique value in $ \R^\infty $, that is $ T $ is injective. 
    Now we define the inverse $ \inv{T} $ as follows. 
    For some $ (a_1, a_2, ...) \in \R^\infty $ we map $ \inv{T} (a_1, a_2, ...) = \varphi = a_1 \varphi_1 + a_2 \varphi_2 + ... $. 
    Then, since each $ \varphi $ is uniquely defined by these polynomials we have shown that $ \inv{T} $ is injective. 
    Since, bot $ T $ and $ \inv{T} $ are injective this completes the proof that $ T $ is invertible, so the two vector spaces are isomorphic.
\end{q}

\section{Exercises 4}
\begin{q}[2]
    No. Consider the following counterexample for a subspace where $ m = 2 $.
    $ x^2 + x $ and $ -x^2 $ are in the subspace. 
    But the vector addition of $ x^2 + x - x^2 = x $ is not in the subspace since it has degree 1.
\end{q}
\begin{q}[3]
    No. Consider the following counterexample for a subspace where $ m = 2 $.
    $ x^2 + x $ and $ -x^2 $ are in the subspace, since they both have degree 2 which is even. 
    But the vector addition of $ x^2 + x - x^2 = x $ is not in the subspace since it has degree 1, which is odd.
\end{q}
\begin{q}[6]
    For the $ \impliedby $ direction if $ p $ and $ p' $ don't have distinct zeros then it means that there exists no $ z \in \C $ such that $ p(z) = 0 = p'(z) $.
    In other words, a point $ z $ such that it is zero and the slope at that point is zero. 
    Then it must be that there is no point $ a \in \C $ such that $ p(z) = (a-z)^m $ for  $ m > 1 $ since by the chain rule $ p'(z) $ will have some form of the term $ m(a-z)^{m-1} $. 
    This, can only happen if $ m > 2 $ and that means that it has repeated zeros. We use the exact same argument to go the other way thus showing the if and only if condiiton. 
\end{q}
\begin{q}[7]
    By 4.15 if a polynomial has a real coefficients and a $ z \in \C $ as a root, the roots come in pairs.
    Since, pairs are well pairs, and the polynomial has odd degree, so odd zeros there is a forever lonely zero left out. 
    This, forever lonely zero must be a real number since if it were complex it would come in a pair. 
\end{q}
\begin{q}[8]
    $ Tp $ is well defined in $ \poly{}{\R} $ since for every polynomial the term $\frac{p-p(3)}{x-3}$ evaluates to a polynomial by our definition of polynomial division. 
    And, when $ x = $ it evaluates to $ p'(3) $ which again is well defined for all polynomials.
    $ T $ is a linear map by closure under addition and scalar multiplication. 
    Observe for $ T(p + q) $ we have 
    $$T (p + q)=\left\{\begin{array}{ll}\frac{p + q -p(3) - q(3)}{x-3} & \text { if } x \neq 3 \\ p^{\prime}(3) + q'(3) & \text { if } x=3\end{array}\right.$$
    Which is exactly equivalent to $ Tp + Tq $. 
    Then for sclar multiplication, 
    $$T \lambda p=\left\{\begin{array}{ll} \lambda \frac{p-p(3)}{x-3} & \text { if } x \neq 3 \\ \lambda p^{\prime}(3) & \text { if } x=3\end{array}\right.$$
    By distributivity for the the top term, and by the power rule for the bottom term. Which are both equivalent to $ \lambda Tp $.
\end{q}
\begin{q}[10]
    Assume for contradiction that one of the coefficients is complex. Then there is some term in $ p $ as $ az^n $ for some $ a \in \C $ and $ n \leq m $. Then, a real number multiplied by a complex number is complex. 
    This is a contradiction and therefore the polynomial must have all real coefficients.  
\end{q}


\end{document}