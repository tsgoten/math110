\documentclass[10pt, twocolumn]{article}
\usepackage{../hw}
\title{Homework 8}
\author{Tarang Srivastava}

\begin{document}
\makechaptertitle

\section{Exercises 5.B}
\begin{q}[1]
    We wish to show that 
    $$ \inv{(I - T)} = I + T + ... + T^{n - 1} $$
    So, we can multiply $ I - T $ to both sides to get 
    $$ I = (I - T)(I + T + ... + T^{n - 1}) $$
    Then, we can distribute and see we get 
    $$ I = I - T + T - T^2 + T^2 + ... - T^{n-1}  + T^{n-1} + T^{n} $$
    After cancelling out all the similar terms we are left with
    $$ I = I + T^{n} = I $$
    Since, $ T^n = 0 $. 
    So, to prove the statement we do the following operations
    $$ I = I + T^{n} = I $$
    $$ I = I - T + T - T^2 + T^2 + ... - T^{n-1}  + T^{n-1} + T^{n} $$
    $$ I = (I - T)(I + T + ... + T^{n - 1}) $$
    Then, we multiply both sides by $ \inv{(I - T)} $ to get 
    $$ \inv{(I - T)} = I + T + ... + T^{n - 1} $$
    as desired.
\end{q}
\begin{q}[2]
    Assume for contradiction that $ \lambda \neq 2 $ and $\lambda \neq 3 $ and $ \lambda \neq 4 $. 
    Then, $ T - 2I $ and $ T - 3I $ and $ T - 4I $ must all be invertible.
    Given,
    $$ (T - 2I)(T - 3I)(T - 4I) = 0 $$
    for all $ v \in V $ such that $ v \neq 0 $ we have that 
    $$ (T - 2I)(T - 3I)(T - 4I)v = 0v = 0 $$
    Then, for one of the values $ T - 2I $ or $ T - 3I $ or $ T - 4I $ one of them maps $ v $ to 0. 
    Since, they are all invertible their null spaces is just $ \{0\} $, but then we have a contradiction since we had that $ v \neq 0 $.
    Therefore, it must be the case that $ \lambda $ is equal to 2, 3 or 4.
\end{q}
\begin{q}[3]
    We proceed directly. 
    Given
    $$ T^2 = I $$
    We get that $ T^2 - I = 0 $ so, 
    $$ (T - I)(T + I) = 0 $$
    Since, $ \lambda \neq -1 $ it must be that $ T + I $ is invertible. 
    Thus, for all non zero $ v $ in $ V $ we have that $ (T + I)v $ is non zero. 
    So it must be that for all $ w \in V $, we have $ (T - I)v = 0 $.
    By definition $ T - I $ is equal to the 0 linear map, so from $ T - I = 0 $ it follows that
    $ T = I $
\end{q}
\begin{q}[4]
    From $ P^2 = P $ we have that $ P^2 - P = 0 $ so it must be that 
    $$ P(P - I) = 0 $$
    So for all $ v \in V $ we have that 
    $$ P(P - I)v = 0v = 0 $$
    Thus
    $$ Pv = 0 \text{ or } Pv = v $$
    To show that $ V = \nul P \oplus \range P $ we first will show that $ \nul P \cap \range P = \emptyspace $.
    Suppose $ v \in \nul P \cap \range P $. 
    Then, $ Pv = 0 $ and $ Pv = v $ it follows directly then that $ v = 0 $, so 
    $$ \nul P \cap \range P = \emptyspace $$
    % NOTE: Ask Alex if it is actually necessary to show this, since $ P $ is an operator. I think it is since $ V $ is infinite. 
    Since, $ P \in \Operator{V} $ we already have that $ \range P \subset V $ and $ \nul P \subset V $. 
    Then for $ v \in \range P $ and $ w \in \nul P $ clearly $ v + w \in V $ so we have that 
    $$ V \supset \range P \oplus \nul P $$
    For the other side, let $ v \in V $ we have that $ Pv = 0 $ or $ Pv = v $. 
    So, $ v \in \range P $ or $ v \in \nul P $ then it follows that clearly for all $ v \in V $ we have that 
    $ v = v + 0 $ or $ v = 0 + v $. 
    So, 
    $$ V \subset \range P \oplus \nul P $$
    Therfore,
    $$ V = \range P \oplus \nul P $$
\end{q}
\begin{q}[8]
    I'm just going to express $ T $ as a matrix in terms of the standard basis of $ \R^2 $, I hope that's okay. 
    So, the goal is that $ T^4 = -1 $ well if we find $ T $ such that it represent an eighth clockwise turn, we are done. 
    Since 4 turns of $ T $ would be equivalent to $ - 1 $. 
    We can construct $ T $ by simply seeing where the basis vectors would be mapped if they were turned by an eigth. 
    So, $ (1, 0) $ would map to $ (\sqrt{2}/2, \sqrt{2}/2)$ and then $ (0, 1) $ would map to $ (\sqrt{2}/2, -\sqrt{2}/2)$. 
    That is basically the definition of $ T $, but in matrix form it is 
    $$ T = \begin{pmatrix}
        \frac{\sqrt{2}}{2} & \frac{\sqrt{2}}{2} \\
        \frac{\sqrt{2}}{2} & -\frac{\sqrt{2}}{2}
        \end{pmatrix} 
    $$
\end{q}
\begin{q}[9]
    If $ p(T)v = 0 $ by properties of polynomials we can then express $ p(T) $ as a polynomial where we factor out the zero. 
    That is for some $ q \in \poly{}{\F} $ and some constant $ \lambda \in \F $, 
    $$ p(T) = q(T) \cdot (T - \lambda) $$
    Then we have that, 
    $ (T - \lambda) v = 0 $, so then by rearranging we get that in fact $ Tv = \lambda v $. 
    So, $ \lambda $ must be an eigenvector of $ T $.
\end{q}
\begin{q}[10]
    Given $ \lambda $ is an eigenvalue of $ T $ we have that $ Tv = \lambda v $ for some eigenvector $ v $. 
    Thus, we know that $ T(Tv) = T \lambda v = \lambda Tv = \lambda^2 v $. We then use this inductive argument to get that $ T^n v = \lambda^n v $.
    Since, $ p(T) $ is of some form as $ p(T)v = a_m T^mv + ... $ we can simply replace each term of $ T^mv $ with $ \lambda^m v$, 
    then we get it is equivalent that $ p(T)v = a_m \lambda^m v + ...  $ which is equivalent to expressing the polynomial for $ \lambda $ 
    That is, $ p(\lambda)v = a_m \lambda v + ... $ so we get 
    $$ p(T)v = p(\lambda)v $$
\end{q}
\begin{q}[11]
    $ \implies $ direction.
    Sice we are working with $ \C $ we have the nice property that 
    $$ p(T) = c(T - \lambda_1 I)...(T - \lambda_m I) $$

    $ \impliedby $ direction. 
    We know that if $ \alpha = p(\lambda) $ for some eigenvalue $ \lambda $ of $ T $ then from Problem 10 we know that 
    $$ p(T) v = p(\lambda) v $$
    where $ v $ is the corresponding eigenvector.
    Then, we directly have that 
    $$ p(T) v = \alpha v $$
    Thus, by definition $ \alpha $ is an eigenvalue of $ p(T) $.
\end{q}

\section{Exercises 5.C}
\begin{q}[1]
    We use the fact that since $ T $ is diagonalizable then 
    $$ V = E(\lambda_1, T) \oplus ... \oplus E(\lambda_m, T) $$
    So, now we just wish to show that $ \nul T $ and $ \range T $ are some combination of these eigenspaces.
    Observe that $ \nul T $ is equivalent to when $ Tv = 0 = 0v $ that is $ \lambda = 0 $. 
    So, if there is an eigenvalue equal to zero then $ \nul T = E(0, T) $.
    Then, observe that $ V $ has a basis consiting of eigenvectors of $ T $, so it suffices to show that if there is a basis of eigenvectors 
    for all $ v \in V $ we have that $ v = a_1u_1 + ... + a_m u_m $ where $ u_i $ is an eigenvector. 
    Well since $ E(\lambda, T ) $ is a subspace of $ V $ it holds that any scalar multiple of $ u_i $ is also in its corresponding eigenspace.
    Therefore, for all $ v \in V $ since we can express it as a sum of $ v = a_1 u_1 + ... + a_m u_m $ and each term is in a corresponding eigenspace 
    then all the $ Tv \neq 0 $ are expressed as that sum which is equivalent to the range of $ T $. 
    Thus, $ V = \nul T \oplus \range T $. 
\end{q}
\begin{q}[3]
    Well (b) and (c) are literally the definition of (a) so showing (b) and (c) are equivalent suffices to show (a) is equivalent.
    For (b) $ \implies $ (c). 
    We have by nullity-rank that 
    $$ \dim V = nullity + rank $$
    We also have that since the null space and range are subspaces of $ V $
    $$ \dim (\nul T + \range T) = nullity + rank - (\nul T \cap \range T) $$
    But since we have that $ V = \nul T + \range T $ We get that
    $$ nullity + rank = nullity + rank - \dim (\nul T \cap \range T) $$
    So that implies that $\dim (\nul T \cap \range T) = 0 $ thus it must be that $ (\nul T \cap \range T) = \emptyspace $. \\
    For (b) $ \impliedby $ (c). 
    We have that 
    $$ \dim (\nul T + \range T) = nullity + rank - (\nul T \cap \range T)$$
    Since, $ (\nul T \cap \range T) = \emptyspace $
    by combining nullity-rank theorem we have 
    $$ \dim (\nul T + \range T) = nullity + rank = \dim V$$
    Thus it must be in a finite dimensional space that
    $$ \nul T +\range T = V $$
\end{q}
\begin{q}[5]
    Given that $ T $ is diagonal we know that by subtracting $ \lambda I $ we still have a matrix that has elements only in the diagonal and zeros everywhere else. 
    So it follows from definition that $ T - \lambda I $ is diagonal as well. After that we can directly apply Problem 1 to show $ \implies $ direction.
    For the $ \impliedby $ direction. 
    Observe that in the null space $ Tv = \lambda v $ so basically $ T $ just scales the vectors by $ \lambda $, 
    so clearly it is that $ T $ is diagonalizable for those vectors. For the vectors in the range 
\end{q}
\begin{q}[16]
    (a)
    We proceed by induction on $ n $. For the base case $ n = 1 $ we have that 
    $$ T(0, 1) = (1, 1) = (F_1, F_2) $$
    Then assume the claim holds for some $ n = k $. That is, 
    $$ T^k(0, 1) = (F_k, F_{k + 1}) $$
    Then we can just compose $ T $ again to get
    $$ T^{k+1}(0, 1) = T(F_k, F_{k + 1}) = (F_{k + 1}, F_k + F_{k + 1}) = (F_{k + 1}, F_{k + 2})$$
    So we showed the inductive step holds and thus the claim holds for all $ n $. \\
    (b) this homework is so fucking long and so fucking boring fuck this shit . I miss discrete math with its fun hw problems
\end{q}

\section{Exercises 6.A}
\begin{q}[11]
    This is a direct application of Cauchy-Schwarz inequality. 
    The statement holds for all positive $ a, b, c, d $ because we set it up with the terms 
    $ \frac{1}{\sqrt{a}} $ and $ \sqrt{a} $ and 
    $ \frac{1}{\sqrt{b}} $ and $ \sqrt{b} $ and 
    $ \frac{1}{\sqrt{c}} $ and $ \sqrt{c} $ and 
    $ \frac{1}{\sqrt{c}} $ and $ \sqrt{c} $. 
    Then we have something of the form
    $$ | 1 + 1 + 1 + 1 |^2 = 16 \leq (a+b+c+d)\left(\frac{1}{a}+\frac{1}{b}+\frac{1}{c}+\frac{1}{d}\right)$$
    which we get from just squaring all the terms before to fit the Cauchy-Schwarz inequality form.
\end{q}


\end{document}