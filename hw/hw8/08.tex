\documentclass[10pt, twocolumn]{article}
\usepackage{../hw}
\title{Homework 8}
\author{Tarang Srivastava}
\begin{document}
\makechaptertitle

\section{Exercises 5.B}
\begin{q}[1]
    We wish to show that 
    $$ \inv{(I - T)} = I + T + ... + T^{n - 1} $$
    So, we can multiply $ I - T $ to both sides to get 
    $$ I = (I - T)(I + T + ... + T^{n - 1}) $$
    Then, we can distribute and see we get 
    $$ I = I - T + T - T^2 + T^2 + ... - T^{n-1}  + T^{n-1} + T^{n} $$
    After cancelling out all the similar terms we are left with
    $$ I = I + T^{n} = I $$
    Since, $ T^n = 0 $. 
    So, to prove the statement we do the following operations
    $$ I = I + T^{n} = I $$
    $$ I = I - T + T - T^2 + T^2 + ... - T^{n-1}  + T^{n-1} + T^{n} $$
    $$ I = (I - T)(I + T + ... + T^{n - 1}) $$
    Then, we multiply both sides by $ \inv{(I - T)} $ to get 
    $$ \inv{(I - T)} = I + T + ... + T^{n - 1} $$
    as desired.
\end{q}
\begin{q}[2]
    Assume for contradiction that $ \lambda \neq 2 $ and $\lambda \neq 3 $ and $ \lambda \neq 4 $. 
    Then, $ T - 2I $ and $ T - 3I $ and $ T - 4I $ must all be invertible.
    Given,
    $$ (T - 2I)(T - 3I)(T - 4I) = 0 $$
    for all $ v \in V $ such that $ v \neq 0 $ we have that 
    $$ (T - 2I)(T - 3I)(T - 4I)v = 0v = 0 $$
    Then, for one of the values $ T - 2I $ or $ T - 3I $ or $ T - 4I $ one of them maps $ v $ to 0. 
    Since, they are all invertible their null spaces is just $ \{0\} $, but then we have a contradiction since we had that $ v \neq 0 $.
    Therefore, it must be the case that $ \lambda $ is equal to 2, 3 or 4.
\end{q}
\begin{q}[3]
    We proceed directly. 
    Given
    $$ T^2 = I $$
    We get that $ T^2 - I = 0 $ so, 
    $$ (T - I)(T + I) = 0 $$
    Since, $ \lambda \neq -1 $ it must be that $ T + I $ is invertible. 
    Thus, for all non zero $ v $ in $ V $ we have that $ (T + I)v $ is non zero. 
    So it must be that for all $ w \in V $, we have $ (T - I)v = 0 $.
    By definition $ T - I $ is equal to the 0 linear map, so from $ T - I = 0 $ it follows that
    $ T = I $
\end{q}
\begin{q}[4]
    From $ P^2 = P $ we have that $ P^2 - P = 0 $ so it must be that 
    $$ P(P - I) = 0 $$
    So for all $ v \in V $ we have that 
    $$ P(P - I)v = 0v = 0 $$
    Thus
    $$ Pv = 0 \text{ or } Pv = v $$
    To show that $ V = \nul P \oplus \range P $ we first will show that $ \nul P \cap \range P = \emptyspace $.
    Suppose $ v \in \nul P \cap \range P $. 
    Then, $ Pv = 0 $ and $ Pv = v $ it follows directly then that $ v = 0 $, so 
    $$ \nul P \cap \range P = \emptyspace $$
    % NOTE: Ask Alex if it is actually necessary to show this, since $ P $ is an operator. I think it is since $ V $ is infinite. 
    Since, $ P \in \Operator{V} $ we already have that $ \range P \subset V $ and $ \nul P \subset V $. 
    Then for $ v \in \range P $ and $ w \in \nul P $ clearly $ v + w \in V $ so we have that 
    $$ V \supset \range P \oplus \nul P $$
    For the other side, let $ v \in V $ we have that $ Pv = 0 $ or $ Pv = v $. 
    So, $ v \in \range P $ or $ v \in \nul P $ then it follows that clearly for all $ v \in V $ we have that 
    $ v = v + 0 $ or $ v = 0 + v $. 
    So, 
    $$ V \subset \range P \oplus \nul P $$
    Therfore,
    $$ V = \range P \oplus \nul P $$
\end{q}


\end{document}