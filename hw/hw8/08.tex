\documentclass[10pt, twocolumn]{article}
\usepackage{../hw}
\title{Homework 8}
\author{Tarang Srivastava}
\begin{document}
\makechaptertitle

\section{Exercises 5.B}
\begin{q}[1]
    We wish to show that 
    $$ \inv{(I - T)} = I + T + ... + T^{n - 1} $$
    So, we can multiply $ I - T $ to both sides to get 
    $$ I = (I - T)(I + T + ... + T^{n - 1}) $$
    Then, we can distribute and see we get 
    $$ I = I - T + T - T^2 + T^2 + ... - T^{n-1}  + T^{n-1} + T^{n} $$
    After cancelling out all the similar terms we are left with
    $$ I = I + T^{n} = I $$
    Since, $ T^n = 0 $. 
    So, to prove the statement we do the following operations
    $$ I = I + T^{n} = I $$
    $$ I = I - T + T - T^2 + T^2 + ... - T^{n-1}  + T^{n-1} + T^{n} $$
    $$ I = (I - T)(I + T + ... + T^{n - 1}) $$
    Then, we multiply both sides by $ \inv{(I - T)} $ to get 
    $$ \inv{(I - T)} = I + T + ... + T^{n - 1} $$
    as desired.
\end{q}

\end{document}