\documentclass[10pt, twocolumn]{article}
\usepackage{../hw}
\title{Homework 11}
\author{Tarang Srivastava}

\begin{document}
\makechaptertitle

\section{Exercises 7.B}

\begin{q}[3]
    Define the linear map $ T $ as follows for each $ z = (z_1, z_2, z_3) \in \C^3 $.
    $$ Tz = (2z_1, 3z_2, 2z_1) $$
    Note the switch up in the last spot. 
    We can check that $ T $ is indeed closed under addition and scalar multiplication, thus a linear map.
    Firstly, $ \lambda = 2 $ is an eigenvalue with the eigenvector $ w = (1, 0, 1) $.
    $$ T(1, 0, 1) = (2, 0, 2) = 2(1, 0, 1) $$
    Similarly, $ \lambda = 3 $ is an eigenvalue with the eigenvector $ w = (0, 1, 0) $.
    $$ T(0, 1, 0) = (0, 3, 0) = 3(0, 1, 0) $$
    Finally, $ (T^2 - 5T + 6I)z \neq 0 $ if $ z_3 \neq 6 z_1 $. 
    So given we have values that are nonzero after that linear map, $ T^2 - 5T + 6I \neq 0 $.
\end{q}

\begin{q}[4]

\end{q}

\end{document}