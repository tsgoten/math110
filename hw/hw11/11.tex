\documentclass[10pt, twocolumn]{article}
\usepackage{../hw}
\title{Homework 11}
\author{Tarang Srivastava}

\begin{document}
\makechaptertitle

\section{Exercises 7.B}

\begin{q}[3]
    Define the linear map $ T $ as follows for each $ z = (z_1, z_2, z_3) \in \C^3 $.
    $$ Tz = (2z_1, 3z_2, 2z_1) $$
    Note the switch up in the last spot. 
    We can check that $ T $ is indeed closed under addition and scalar multiplication, thus a linear map.
    Firstly, $ \lambda = 2 $ is an eigenvalue with the eigenvector $ w = (1, 0, 1) $.
    $$ T(1, 0, 1) = (2, 0, 2) = 2(1, 0, 1) $$
    Similarly, $ \lambda = 3 $ is an eigenvalue with the eigenvector $ w = (0, 1, 0) $.
    $$ T(0, 1, 0) = (0, 3, 0) = 3(0, 1, 0) $$
    Finally, $ (T^2 - 5T + 6I)z \neq 0 $ if $ z_3 \neq 6 z_1 $. 
    So given we have values that are nonzero after that linear map, $ T^2 - 5T + 6I \neq 0 $.
\end{q}

\begin{q}[4]
    Given that $ \F = \C $ by the complex spectral theorem $ T $ is normal if and only if $ T $ has a diagonal matrix with respect to some orthonormal basis.
    Since $ T $ is diagonalizable if and only if $ V $ is equal to the sum of the eigenspaces of distinct eigenvalues, we have shown the last case.
    Also by the complex spectral theorem $ T $ is normal if and only if $ V $ has an orthonormal basis consisting of eigenvectors of $ T $.
    That completes the proof. 
    (Ask Alex if there's some part that guarantees that we have distinct eigenvalues, but it is not that difficult to show that if all the basis eigenvectors are orthogonal they must be associated with distinct eigenvalues.)
\end{q}

\begin{q}[6]
    $ \implies $ direction follows easily from 7.13, if $ T $ is self-adjoint then all the eigenvalues are real. \\
    $ \impliedby $ Given $ T $ is normal then $ T^* $ and $ T $ have the same eigenvectors 
    and for the corresponding eigenvalue $ \lambda $ for $ T $, $ T ^* $ has the associated eigenvalue $ \overline{\lambda} $ for the same eigenvector.
    If all the eigenvalues are real $ \lambda = \overline{\lambda} $, so $ T $ and $ T^* $ have the same eigenvalues and eigenvectors.
    By the complex spectral theorem $ V $ has an orthogonal basis consisting of eigenvectors of $ T $.
    Denote these basis eigenvectors as $ e_1, ..., e_n $.
    Let $ v \in V $, then for some $ a_1, ..., a_n \in \C $
    \begin{align*}
    Tv &= a_1T(e_1) + ... + a_n T(e_n) \\
    &= a_1 \lambda_1 e_1 + ... + a_n \lambda_n e_n \\
    &= a_1T^*(e_1) + ... + a_n T^*(e_n) \\
    Tv &= T^*v
    \end{align*}
    For all $ v \in V $ so $ T = T^* $, therefore $ T $ is self adjoint.
\end{q}

\end{document}