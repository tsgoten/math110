\documentclass[10pt, twocolumn]{article}
\author{Tarang Srivastava}
\usepackage{amsmath, amssymb, amsthm, commath, chngcntr, enumitem, multirow, thmtools, xcolor}
\usepackage{graphicx}
\usepackage[margin=.25in]{geometry}
\setlength{\columnsep}{.5in}
\newcommand{\C}{\mathbb{C}}
\newcommand{\question}[1]{\textcolor{blue}{#1} \\}
\newcommand{\R}{\mathbb{R}}
\newcommand{\F}{\mathbb{F}}
\newcommand{\N}{\mathbb{N}}
\newcommand{\poly}{\mathcal{P}_{m}\left(\F\right)}
\newcommand{\vspan}[1]{\text{span}\left(#1\right)}
\newcommand{\inv}[1]{#1^{-1}}
\newcommand{\todo}[1]{\textcolor{red}{TODO: #1} \\}
\newcommand{\makechaptertitle}[1]{
\begin{center}
	\begin{large}
		#1
	\end{large}
	\begin{small}
		\\Tarang Srivastava
	\end{small}
\end{center}
}
\declaretheoremstyle[
spaceabove=\topsep, spacebelow=\topsep,
headfont=\normalfont\bfseries,
notefont=\bfseries, notebraces={}{},
bodyfont=\normalfont,
postheadspace=0.5em,
name={\ignorespaces},
numbered=no,
headpunct=:]
{mystyle}
\declaretheorem[style=mystyle]{q}

\begin{document}
	
\makechaptertitle{Math 110 Homework 3}

\section{Exercises 2.C}
\begin{q}[Problem 1]
	Since, $ U $ is a subspace there exists a basis $ B $ such that $ \abs{B} = \dim U = \dim V$.
	We have that $ B $ is a linearly independent list in $ V $ and of the right length therefore it must be a basis of $ V $.
	So, $ U = \vspan{B} = V $.
\end{q}
\begin{q}[Problem 7]
	
\end{q}
\begin{q}[Problem 8]
	
\end{q}
\begin{q}[Problem 9]
	
\end{q}
\begin{q}[Problem 10]
	We begin by showing that $ p_0,..., p_m $ is a linearly independent list. 	
	Observe, that any linear combination of $ p_0,..., p_m $ is a polynomial of degree $ m $. 
	So, the only polynomial of degree $ m $ that is the $ 0 $ polynomial is the one with all zero coefficients. 
	Thus, $ p_0,..., p_m $ is a linearly independent list. 
	Since, it is of the right length $ m + 1 $, and for all $ 0 \leq i \leq m $, $ p_i $ is in $ \poly $.
	Then, it must be a basis for $ \poly $.
\end{q}
\begin{q}
	Assume for contradiction that $ U \cap W = \{0\}$.
	Then, $ U + W $ is a direct sum, and $ U, W \subset V $. 
	So, it must be that $ U \oplus W = V $. 
	Thereofre, $ \dim V = 9 = \dim W + \dim U - \dim W \cap U = 5 + 5 - 0 $. 
	Here we have a contradiction, thus, $ U \cap W \neq \{0\} $.
\end{q}

\end{document}