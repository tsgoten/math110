\documentclass[10pt, twocolumn]{article}
\author{Tarang Srivastava}
\usepackage{amsmath, amssymb, amsthm, commath, chngcntr, enumitem, multirow, thmtools, xcolor}
\usepackage{graphicx}
\usepackage[margin=.25in]{geometry}
\setlength{\columnsep}{.5in}
\newcommand{\C}{\mathbb{C}}
\newcommand{\question}[1]{\textcolor{blue}{#1} \\}
\newcommand{\R}{\mathbb{R}}
\newcommand{\F}{\mathbb{F}}
\newcommand{\N}{\mathbb{N}}
\newcommand{\poly}[2]{\mathcal{P}_{#1}\left(#2\right)}
\newcommand{\vspan}[1]{\text{span}\left(#1\right)}
\newcommand{\inv}[1]{#1^{-1}}
\newcommand{\todo}[1]{\textcolor{red}{TODO: #1} \\}
\newcommand{\nul}{\text{null }}
\newcommand{\range}{\text{range }}
\newcommand{\makechaptertitle}[1]{
\begin{center}
	\begin{large}
		#1
	\end{large}
	\begin{small}
		\\Tarang Srivastava
	\end{small}
\end{center}
}
\declaretheoremstyle[
spaceabove=\topsep, spacebelow=\topsep,
headfont=\normalfont\bfseries,
notefont=\bfseries, notebraces={Problem }{},
bodyfont=\normalfont,
postheadspace=0.5em,
name={\ignorespaces},
numbered=no,
headpunct=:]
{mystyle}
\declaretheorem[style=mystyle]{q}

\begin{document}
	
\makechaptertitle{Math 110 Homework 3}

\section{Exercises 2.C}
\begin{q}[1]
	Since, $ U $ is a subspace there exists a basis $ B $ such that $ \abs{B} = \dim U = \dim V$.
	We have that $ B $ is a linearly independent list in $ V $ and of the right length therefore it must be a basis of $ V $.
	So, $ U = \vspan{B} = V $.
\end{q}
\begin{q}[7]
	We have the obvious polynomial $ 1 $, which will always be the same. 
	Then, we can construct a polynomial $ (x - 2)(x-5)(x-6) $ which is always 0 for $2, 5, 6 $. 
	Additionally, we can multiply by $ x $ to get $ x(x-2)(x-5)(x-6) $ which is also always equal to 0 for the following values. 
	Now, we have to show that they are linearly independent, which follows from the fact that they all have different degrees of $ 0, 3, 4 $ respectively. 
	So, they are linearly independent. Therefore, we have a basis for $ U $. 
	We use a similar argument from HW2.A.17 where we were able to show that the dimension is one less, when we restricted one value. 
	So, if we restrict the three terms it has at most dimension from 5 to 3. So, we only need to show three vectors to form a basis.
	Now, if we add in $ x, x^2 $ we can uniquely express all the elements in $ \poly{4}{\F} $ as a linear combination of those vectors, 
	thus it forms a basis for $ \poly{4}{\F} $.
	We just wish to find a subspace where $ p(5) \neq p(6) $ or $ p(6)\neq p(2) $ or $ p(2) \neq p(5) $ with this restriction we find every polynomial not in $ U $. 
	Thus, the intersection is as desired and $ U \oplus W \poly{4}{\F} $.
\end{q}
\begin{q}[8]
	Observe that all polynomials that satisfy the condition that the integeral form -1 to 1 of the polynomial is zero 
	must be odd, that is $ f(-x) = -f(x) $. 
	We also know that odd polynomials only have odd degree terms. 
	That is, $ U \in \vspan{x, x^3} $.
	Then, we just add the remaining polynomials from the standard basis to get a basis for $ \poly{4}{\R} $.
	So, we define $ W = \vspan{1, x^2, x^4} $ such that they form a direct sum. 
	We know this is the case since they are linearly independent.
	We also know the intersectin is clearly $ \{0\} $.
	Additionally, the sum of the standard basis is the definition for every $ p \in \poly{4}{\R} $.
	Namely $ W $ is the even polynomials.
\end{q}
\begin{q}[9]
	We know that if $ W $ is an infinite dimensional set then there exist $ n $ sized independent lists for any positive integer $ n $. 
	That is, $ w_1, ..., w_n $.
	This shows that we can have a valid linear map $ T: V \to W $ by 3.5. 
	Suppose we have that $ a_1 T_1 + ... + a_n T_n = 0 $. 
	Then we can add the linear pairs to get zero linear pair. 
	That is, 
	$ (a_1 T_1 + ... + a_n T_n)(v) = 0 $
	Then we select $ T_i v $ such that $ T_i v = w_i $. 
	So, we then have a linear combination $ a_1 w_1 + ... + a_n w_n = 0 $, so the previous linear combinaiton is zero as well.
	This, holds for all real numbers $ n$ so by the previous argument there are  infinite linear maps since there are infinite linearly independent lists in $ \mathcal{L} (V, W) $.
\end{q}
\begin{q}[10]
	We begin by showing that $ p_0,..., p_m $ is a linearly independent list. 	
	Observe, that any linear combination of $ p_0,..., p_m $ is a polynomial of degree $ m $. 
	So, the only polynomial of degree $ m $ that is the $ 0 $ polynomial is the one with all zero coefficients. 
	Thus, $ p_0,..., p_m $ is a linearly independent list. 
	Since, it is of the right length $ m + 1 $, and for all $ 0 \leq i \leq m $, $ p_i $ is in $ \poly{m}{\F} $.
	Then, it must be a basis for $ \poly{m}{\F} $.
\end{q}
\begin{q}[12]
	Assume for contradiction that $ U \cap W = \{0\}$.
	Then, $ U + W $ is a direct sum, and $ U, W \subset V $. 
	So, it must be that $ U \oplus W = V $. 
	Thereofre, $ \dim V = 9 = \dim W + \dim U - \dim W \cap U = 5 + 5 - 0 $. 
	Here we have a contradiction, thus, $ U \cap W \neq \{0\} $.
\end{q}
\begin{q}[13]
	We know that $ \dim \C^6 = \dim U + \dim W - \dim (U \cap W) $. 
	So the dimension of the intersection is $ \dim (U \cap W) = 4 + 4 - 6 = 2 $.
	There exists a basis $ B $ such that $ |B| = 2 $. 
	Denote the two vectors in the basis of $ B $ as $ u, w \in B $. 
	Since, $ u, w $ form a linearly independent list, then they are not scalar multiples of each other.
\end{q}
\begin{q}[14]
	Suppose that all the subspaces of $ V $ are disjoint.
	That is, for all $ i, j \in [1, m] $ it is the case that $ U_i \cap U_j = \{0\} $. 
	Then, the dimension of $ U_1 + ... + U_m $ is just the sum of the individual dimensions. 
	That is, $ \dim (U_1 + ... + U_m) =  \dim U_1 + ... + \dim U_m $. 
	Now, consider that the subspaces are not disjoint. 
	Then, we will have to subtract the intersectin of the subspaces from our previous result.
	So, the valuse for the dimension will be less. 
	Thus, we get the inequality
	$ \dim (U_1 + ... + U_m) \leq  \dim U_1 + ... + \dim U_m $.  
	It follows, that the sum is finite dimensional, since it is a finite sum of finite dimensions. 
\end{q}
\section{Exercises 3.A}
\begin{q}[3]
	Observe that that for the standard basis $ \F^n $, 
	we can arrive at the following scalars 
	\begin{align*}
		T(1, ..., 0) &= (A_{1, 1}, A_{2, 1}, ..., A_{m, 1}) \\
		& \vdots \\
		T(0, ..., 1) &= (A_{1, n}, A_{2, n}, ..., A_{m, n})
	\end{align*}
	Then, by theorem 3.5 there exists a linear map $ T: V \to W $. 
	Since, we can multiply by some scalar $ x_i $ for $ i = 1, ..., n $ and then add those such vectors. 
	We can claim that, 
	$$ T(x_1, ..., x_n ) =  (A_{1, 1}x_1 + ... + A_{1, n}x_n, ..., A_{m, 1}x_1 + ...+ A_{m, n} x_n) $$
	by the additivity of linear maps. 
\end{q}
\begin{q}[4]
	To show linear independence consider the aribitrary linear combination equal to zero
	$$ 0 = a_1v_1 + ... + a_m v_m $$
	By the additivity and homogeneity properties we have 
	$$ T(0) = 0 = T(a_1v_1 + ... + a_m v_m) = a_1Tv_1 + ... + a_mTv_m $$
	Then, since we know $ Tv_1, ..., Tv_m $ is linearly independent it must be the case that all the coefficients are zero. 
	That is, $ a_1 = ... = a_m = 0 $. 
	So, we have shown that $ v_1, ..., v_m $ are linearly independent. 
\end{q}
\begin{q}[8]
	We will define $ \varphi: \R^2 \to \R $ as follows, 
	\[ \varphi (x, y) = \sqrt{x^2 + y^2} \]
	Notice that $ \varphi(a(x, y)) = \varphi(ax, ay) = \sqrt{a^2x^2 + a^2y^2} = a \sqrt{x^2 + y^2} $ by the definition of scalar multplication on $ \R^2 $.
	Also, $\varphi(x, y) = a\sqrt{x^2 + y^2}$ showing that $ \varphi $ satisfies the condition.
	Consider the vectors $ (1, 0) $ and $ (0, 1) $. 
	We know that $ (1, 0) + (0, 1) = (1, 1) $. 
	But $ \varphi $ does not satisfy additivity since 
	$ \varphi (1, 0) = 1 = \varphi (0, 1) $, but $ \varphi(1, 1) = \sqrt{2} $ and we know that $ 2 \neq \sqrt{2} $.
\end{q}
\begin{q}[9]
	We will define $ \varphi: \C \to \C $ as follows, 
	$$ \varphi(a + bi) = a $$ 
	We can show that for some $ w = w_1 + w_2i $ and $ z = z_1+ z_2i $, our function respects additivity.
	$$ \varphi(w + z) = \varphi(w_1 + z_1 + (w_2 + z_2)i) = w_1 + z_1 = \varphi(w) + \varphi(z) $$
	Then, if we select the scalar $ i \in \C $ we observe that $ \varphi(iz) = -z_2 \neq i\varphi(z) = z_1i $. 
	Thus, it does not satisfy homogeneity. 
\end{q}
\begin{q}[10]
	Let $ u \in U $ be a vector such that it does not map to 0, that is $ Su \neq 0 $. 
	Now, select a $ v \in V $ such that $ v \not\in U $. 
	We can choose such a $ v $, because $ U \neq V $ and $ U $ is a subspace of $ V $. 
	We know that $ v + u \not\in U $, since if it were, then there is a $ w \in U $ such that $ w = v + u $, 
	and $ w - u = v $. 
	Since, $ U $ is closed under addition that would mean that $ v \in U $ which is a contradiction. 
	Finally, if we apply the linear map we get that 
	$$ T(u + v)	 = 0 \neq Tu + Tv = Su + 0 $$
	Since, we initially stated that that $ Su \neq 0 $, it must be the case that $ T $ does not respect additivity and is not a linear map. 
\end{q}
\begin{q}[11]
	Since, $ V $ is finite dimensional and $ U $ is a subspace of $ V $ we know there exists a subspace $ U' $
	such that $ U \oplus U' = V $. 
	Let $ R \in \mathcal{L}(U', W) $ be a linear map from $U'$ to $ W $. 
	Suppose for some $ u \in U $ and $ u' \in U'$ we define the linear map $ T $ as
	$ Tu = Su $ and $ Tu' = Ru' $. 
	Then for any $ v \in V $ there is a sum of $ u + u' = v $, so  $ Tv = Tu + Tu' = Su + Ru' $. 
	Since, every $ v $ has a linar map $ Tv $, we have found a $ T $ that satsifies our condition.
\end{q}
\begin{q}[12]
	We know that if $ W $ is an infinite dimensional set then there exist $ n $ sized independent lists for any positive integer $ n $. 
	That is, $ w_1, ..., w_n $.
	This shows that we can have a valid linear map $ T: V \to W $ by 3.5. 
	Suppose we have that $ a_1 T_1 + ... + a_n T_n = 0 $. 
	Then we can add the linear pairs to get zero linear pair. 
	That is, 
	$ (a_1 T_1 + ... + a_n T_n)(v) = 0 $
	Then we select $ T_i v $ such that $ T_i v = w_i $. 
	So, we then have a linear combination $ a_1 w_1 + ... + a_n w_n = 0 $, so the previous linear combinaiton is zero as well.
	This, holds for all real numbers $ n$ so by the previous argument there are  infinite linear maps since there are infinite linearly independent lists in $ \mathcal{L} (V, W) $.

\end{q}
\begin{q}[14]
	Let $ V = \R^2 $, which we know has dimension 2, which is $ \geq 2 $. 
	The let $ T $ be a linear map that for $ v \in \R^2 $ such that $ v = (x, y)$, 
	we will have $ Tv = (y, x) $. 
	We know this satisfies additivity and homogeneity since any $ (y, x) \in \R^2 $ is in a Vector space which is closed under vector addition and scalar multiplication.
	Additoinally we know from teh notes that $ (2x, 2y) $ is also a valid vector space. 
	Therefore, we can construct a linear map $ S $
	such that $ Sv = (2x, 2y) $. 
	Then. note by the previous argument that $S$ must satisfy additivity and homogeneity. 
	We then see that $ S(Tv) = S(y, x) = (2y, 2x) $, 
	Additionally, we can contruct the opposite argument that 
	$ T(Sv) = T(y, x) = (y, x)$. 
	So, if we multiply the last expression by 2 we get
	$ 2T(Sv) 2= T(y, x) = 2(y, x) = (2y, 2x) $
	So we equate first two expressions in each equation to get 
	$ S(Tv) = 2T(Sv) $ 
	So, we have shown that there exist linear maps $ S, T $ such that $ ST \neq TS $. 
\end{q}

\section{Exercises 3.B}
\begin{q}[1]
	Let $ T $ be a linear map from $ \R^5 \to \R^5 $ such that for some $ x \in \R^5 $, 
	where $ x = (x_1, x_2, x_3, x_4, x_5) $.
	$$ Tx = (0, 0, 0, x_4, x_5) $$
	First, we wish to show that $ T $ is in fact a linear map. 
	For additivity, we know 
	$$ T(x + y) = T(x_1 + y_1, ..., x_5 + y_5) = (0, 0, 0, x_4 + y_4, x_5 + y_5)$$
	For the other part we have that 
	$$ T(x) + T(y) = (0, 0, 0, x_4, x_5) + (0, 0, 0, y_4, y_5) $$ 
	$$ = (0, 0, 0, x_4 + y_4, x_5 + y_5) $$
	Therefore, additivity holds.
	For homogeneity we simply show again that 
	$$ T(cx) = T(cx_1, ..., cx^5) = (0, 0, 0, cx_4, cx_5) $$
	So, then for the other part
	$$ cT(x) = c (0, 0, 0, x_4, x_5) =(0, 0, 0, cx_4, cx_5) $$
	Therefore, showing homogeneity holds as well. 
	So, $ T $ is a valid linear map.
	It follows quite easily now that the null space is all the elements with zeros in the last two indices. 
	That is $$ \nul T = \vspan{(1, 0, 0, 0, 0 ), (0, 1, 0, 0, 0), (0, 0, 1, 0, 0)} $$ 
	These, are all elements of the standard basis of $ \R^5 $ so they are linearly independent and therefore form a basis for $ \nul T $.
	So, we have shown that the null space has dimension 3. We know $ \R^5 $ has dimension 5, so it follows that the range has dimension 2, 
	but again that is not very hard to show, but not necessary in this case by 3.22.
\end{q}
\begin{q}[2]
	Observe that $ S, T $ are both  maps from $ V \to V $ so performing $ (ST)^2 = STST $ is completely valid.
	For any $ v \in V $ if $ Tv = 0 $ or $ Sv = 0 $ we are immediately done, since $ S $ and $ T $ are linear maps and must always map 0 to 0, 
	so after the chain of linear maps we will have 0 at the end. 
	To the more interesting part, we first get that $ Ts \neq 0 $. 
	We are concerned about when $ S(Ts) \in \text{range } S$, because we showed the other case just leads to 0.
	Then, since $ \text{ range} S \subset \nul T $ it must be then that $ T(S(Ts)) = 0 $. 
	So, we have shown that $ (ST)^2 = 0 $, since it maps all $ v $ to 0. 
\end{q}
\begin{q}[5]
	Let $ T $ be the linear map that takes the first two indices and moves them to the last two, and sets the first two to zero. 
	That is, for $ x \in \R^4 $ such that $ x = (x_1, x_2, x_3, x_4) $ we have that 
	$$ Tx = (0, 0, x_1, x_2) $$ 
	By the exact same arguments as \textbf{Problem 1} we know that $ T $ is a valid linear map. 
	Observe, that the range of $ T $ is given by the standard basis vectors $ (0, 0, 1, 0), (0, 0, 0, 1) $. 
	So, it is that $ \range T = \vspan{(0, 0, 1, 0), (0, 0, 0, 1) } $. 
	We also know that these are the exact basis for the null space, since any $ v \in \vspan{(0, 0, 1, 0), (0, 0, 0, 1)} $ 
	Already, has zeros in the first two indices, so after the linear map it will have zeros in all the indices. Therefore it is in the $ \nul T $. 
	Since, the same basis define the following it must be that $ \nul T = \range T $. 
\end{q}
\begin{q}[6]
	We know that $$ \dim V = \dim \range	T + \dim \nul T $$
	Assume for contradiction that $ \dim \range	T = \dim \nul T $, 
	so we can make the following substitutions and get 
	$$ 5 = 2 \dim \range T $$
	Here we have a contradiction since it claims that $ \dim \range T = \frac{5}{2} $, 
	but a dimensions is the length of a basis so must always be a natural number. 
	So, it must be that for $ \R^5 $ the range and null space always have unequal dimensions.
\end{q}

\end{document}