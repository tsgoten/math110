\documentclass[10pt, twocolumn]{article}
\author{Tarang Srivastava}
\usepackage{amsmath, amssymb, amsthm, commath, chngcntr, enumitem, multirow, thmtools, xcolor}
\usepackage{graphicx}
\usepackage[margin=.25in]{geometry}
\setlength{\columnsep}{.5in}
\newcommand{\C}{\mathbb{C}}
\newcommand{\question}[1]{\textcolor{blue}{#1} \\}
\newcommand{\R}{\mathbb{R}}
\newcommand{\F}{\mathbb{F}}
\newcommand{\N}{\mathbb{N}}
\newcommand{\poly}{\mathcal{P}_{m}\left(\F\right)}
\newcommand{\vspan}[1]{\text{span}\left(#1\right)}
\newcommand{\inv}[1]{#1^{-1}}
\newcommand{\todo}[1]{\textcolor{red}{TODO: #1} \\}
\newcommand{\makechaptertitle}[1]{
\begin{center}
	\begin{large}
		#1
	\end{large}
	\begin{small}
		\\Tarang Srivastava
	\end{small}
\end{center}
}
\declaretheoremstyle[
spaceabove=\topsep, spacebelow=\topsep,
headfont=\normalfont\bfseries,
notefont=\bfseries, notebraces={}{},
bodyfont=\normalfont,
postheadspace=0.5em,
name={\ignorespaces},
numbered=no,
headpunct=:]
{mystyle}
\declaretheorem[style=mystyle]{q}

\begin{document}
	
\makechaptertitle{Math 110 Homework 3}

\section{Exercises 2.C}
\begin{q}[Problem 1]
	Since, $ U $ is a subspace there exists a basis $ B $ such that $ \abs{B} = \dim U = \dim V$.
	We have that $ B $ is a linearly independent list in $ V $ and of the right length therefore it must be a basis of $ V $.
	So, $ U = \vspan{B} = V $.
\end{q}
\begin{q}[Problem 7]
	
\end{q}
\begin{q}[Problem 8]
	
\end{q}
\begin{q}[Problem 9]
	
\end{q}
\begin{q}[Problem 10]
	We begin by showing that $ p_0,..., p_m $ is a linearly independent list. 	
	Observe, that any linear combination of $ p_0,..., p_m $ is a polynomial of degree $ m $. 
	So, the only polynomial of degree $ m $ that is the $ 0 $ polynomial is the one with all zero coefficients. 
	Thus, $ p_0,..., p_m $ is a linearly independent list. 
	Since, it is of the right length $ m + 1 $, and for all $ 0 \leq i \leq m $, $ p_i $ is in $ \poly $.
	Then, it must be a basis for $ \poly $.
\end{q}
\begin{q}[Problem 12]
	Assume for contradiction that $ U \cap W = \{0\}$.
	Then, $ U + W $ is a direct sum, and $ U, W \subset V $. 
	So, it must be that $ U \oplus W = V $. 
	Thereofre, $ \dim V = 9 = \dim W + \dim U - \dim W \cap U = 5 + 5 - 0 $. 
	Here we have a contradiction, thus, $ U \cap W \neq \{0\} $.
\end{q}
\begin{q}[Problem 13]
	We know that $ \dim \C^6 = \dim U + \dim W - \dim (U \cap W) $. 
	So the dimension of the intersection is $ \dim (U \cap W) = 4 + 4 - 6 = 2 $.
	There exists a basis $ B $ such that $ |B| = 2 $. 
	Denote the two vectors in the basis of $ B $ as $ u, w \in B $. 
	Since, $ u, w $ form a linearly independent list, then they are not scalar multiples of each other.
\end{q}
\begin{q}[Problem 14]
	Suppose that all the subspaces of $ V $ are disjoint.
	That is, for all $ i, j \in [1, m] $ it is the case that $ U_i \cap U_j = \{0\} $. 
	Then, the dimension of $ U_1 + ... + U_m $ is just the sum of the individual dimensions. 
	That is, $ \dim (U_1 + ... + U_m) =  \dim U_1 + ... + \dim U_m $. 
	Now, consider that the subspaces are not disjoint. 
	Then, we will have to subtract the intersectin of the subspaces from our previous result.
	So, the valuse for the dimension will be less. 
	Thus, we get the inequality
	$ \dim (U_1 + ... + U_m) \leq  \dim U_1 + ... + \dim U_m $.  
	It follows, that the sum is finite dimensional, since it is a finite sum of finite dimensions. 
\end{q}

\end{document}