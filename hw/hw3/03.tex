\documentclass[10pt, twocolumn]{article}
\author{Tarang Srivastava}
\usepackage{amsmath, amssymb, amsthm, commath, chngcntr, enumitem, multirow, thmtools, xcolor}
\usepackage{graphicx}
\usepackage[margin=.25in]{geometry}
\setlength{\columnsep}{.5in}
\newcommand{\C}{\mathbb{C}}
\newcommand{\question}[1]{\textcolor{blue}{#1} \\}
\newcommand{\R}{\mathbb{R}}
\newcommand{\F}{\mathbb{F}}
\newcommand{\N}{\mathbb{N}}
\newcommand{\poly}{\mathcal{P}_{m}\left(\F\right)}
\newcommand{\vspan}[1]{\text{span}\left(#1\right)}
\newcommand{\inv}[1]{#1^{-1}}
\newcommand{\todo}[1]{\textcolor{red}{TODO: #1} \\}
\newcommand{\makechaptertitle}[1]{
\begin{center}
	\begin{large}
		#1
	\end{large}
	\begin{small}
		\\Tarang Srivastava
	\end{small}
\end{center}
}
\declaretheoremstyle[
spaceabove=\topsep, spacebelow=\topsep,
headfont=\normalfont\bfseries,
notefont=\bfseries, notebraces={Problem }{},
bodyfont=\normalfont,
postheadspace=0.5em,
name={\ignorespaces},
numbered=no,
headpunct=:]
{mystyle}
\declaretheorem[style=mystyle]{q}

\begin{document}
	
\makechaptertitle{Math 110 Homework 3}

\section{Exercises 2.C}
\begin{q}[1]
	Since, $ U $ is a subspace there exists a basis $ B $ such that $ \abs{B} = \dim U = \dim V$.
	We have that $ B $ is a linearly independent list in $ V $ and of the right length therefore it must be a basis of $ V $.
	So, $ U = \vspan{B} = V $.
\end{q}
\begin{q}[7]
	
\end{q}
\begin{q}[8]
	
\end{q}
\begin{q}[9]
	
\end{q}
\begin{q}[10]
	We begin by showing that $ p_0,..., p_m $ is a linearly independent list. 	
	Observe, that any linear combination of $ p_0,..., p_m $ is a polynomial of degree $ m $. 
	So, the only polynomial of degree $ m $ that is the $ 0 $ polynomial is the one with all zero coefficients. 
	Thus, $ p_0,..., p_m $ is a linearly independent list. 
	Since, it is of the right length $ m + 1 $, and for all $ 0 \leq i \leq m $, $ p_i $ is in $ \poly $.
	Then, it must be a basis for $ \poly $.
\end{q}
\begin{q}[12]
	Assume for contradiction that $ U \cap W = \{0\}$.
	Then, $ U + W $ is a direct sum, and $ U, W \subset V $. 
	So, it must be that $ U \oplus W = V $. 
	Thereofre, $ \dim V = 9 = \dim W + \dim U - \dim W \cap U = 5 + 5 - 0 $. 
	Here we have a contradiction, thus, $ U \cap W \neq \{0\} $.
\end{q}
\begin{q}[13]
	We know that $ \dim \C^6 = \dim U + \dim W - \dim (U \cap W) $. 
	So the dimension of the intersection is $ \dim (U \cap W) = 4 + 4 - 6 = 2 $.
	There exists a basis $ B $ such that $ |B| = 2 $. 
	Denote the two vectors in the basis of $ B $ as $ u, w \in B $. 
	Since, $ u, w $ form a linearly independent list, then they are not scalar multiples of each other.
\end{q}
\begin{q}[14]
	Suppose that all the subspaces of $ V $ are disjoint.
	That is, for all $ i, j \in [1, m] $ it is the case that $ U_i \cap U_j = \{0\} $. 
	Then, the dimension of $ U_1 + ... + U_m $ is just the sum of the individual dimensions. 
	That is, $ \dim (U_1 + ... + U_m) =  \dim U_1 + ... + \dim U_m $. 
	Now, consider that the subspaces are not disjoint. 
	Then, we will have to subtract the intersectin of the subspaces from our previous result.
	So, the valuse for the dimension will be less. 
	Thus, we get the inequality
	$ \dim (U_1 + ... + U_m) \leq  \dim U_1 + ... + \dim U_m $.  
	It follows, that the sum is finite dimensional, since it is a finite sum of finite dimensions. 
\end{q}
\section{Exercises 3.A}
\begin{q}[3]
	Observe that that for the standard basis $ \F^n $, 
	we can arrive at the following scalars 
	\begin{align*}
		T(1, ..., 0) &= (A_{1, 1}, A_{2, 1}, ..., A_{m, 1}) \\
		& \vdots \\
		T(0, ..., 1) &= (A_{1, n}, A_{2, n}, ..., A_{m, n})
	\end{align*}
	Then, by theorem 3.5 there exists a linear map $ T: V \to W $. 
	Since, we can multiply by some scalar $ x_i $ for $ i = 1, ..., n $ and then add those such vectors. 
	We can claim that, 
	$$ T(x_1, ..., x_n ) =  (A_{1, 1}x_1 + ... + A_{1, n}x_n, ..., A_{m, 1}x_1 + ...+ A_{m, n} x_n) $$
	by the additivity of linear maps. 
\end{q}
\begin{q}[4]
	To show linear independence consider the aribitrary linear combination equal to zero
	$$ 0 = a_1v_1 + ... + a_m v_m $$
	By the additivity and homogeneity properties we have 
	$$ T(0) = 0 = T(a_1v_1 + ... + a_m v_m) = a_1Tv_1 + ... + a_mTv_m $$
	Then, since we know $ Tv_1, ..., Tv_m $ is linearly independent it must be the case that all the coefficients are zero. 
	That is, $ a_1 = ... = a_m = 0 $. 
	So, we have shown that $ v_1, ..., v_m $ are linearly independent. 
\end{q}
\begin{q}[8]
	We will define $ \varphi: \R^2 \to \R $ as follows, 
	\[ \varphi (x, y) = \sqrt{x^2 + y^2} \]
	Notice that $ \varphi(a(x, y)) = \varphi(ax, ay) = \sqrt{a^2x^2 + a^2y^2} = a \sqrt{x^2 + y^2} $ by the definition of scalar multplication on $ \R^2 $.
	Also, $\varphi(x, y) = a\sqrt{x^2 + y^2}$ showing that $ \varphi $ satisfies the condition.
	Consider the vectors $ (1, 0) $ and $ (0, 1) $. 
	We know that $ (1, 0) + (0, 1) = (1, 1) $. 
	But $ \varphi $ does not satisfy additivity since 
	$ \varphi (1, 0) = 1 = \varphi (0, 1) $, but $ \varphi(1, 1) = \sqrt{2} $ and we know that $ 2 \neq \sqrt{2} $.
\end{q}
\begin{q}[9]
	We will define $ \varphi: \C \to \C $ as follows, 
	$$ \varphi(a + bi) = a $$ 
	We can show that for some $ w = w_1 + w_2i $ and $ z = z_1+ z_2i $, our function respects additivity.
	$$ \varphi(w + z) = \varphi(w_1 + z_1 + (w_2 + z_2)i) = w_1 + z_1 = \varphi(w) + \varphi(z) $$
	Then, if we select the scalar $ i \in \C $ we observe that $ \varphi(iz) = -z_2 \neq i\varphi(z) = z_1i $. 
	Thus, it does not satisfy homogeneity. 
\end{q}
\begin{q}[10]
	Let $ u \in U $ be a vector such that it does not map to 0, that is $ Su \neq 0 $. 
	Now, select a $ v \in V $ such that $ v \not\in U $. 
	We can choose such a $ v $, because $ U \neq V $ and $ U $ is a subspace of $ V $. 
	We know that $ v + u \not\in U $, since if it were, then there is a $ w \in U $ such that $ w = v + u $, 
	and $ w - u = v $. 
	Since, $ U $ is closed under addition that would mean that $ v \in U $ which is a contradiction. 
	Finally, if we apply the linear map we get that 
	$$ T(u + v)	 = 0 \neq Tu + Tv = Su + 0 $$
	Since, we initially stated that that $ Su \neq 0 $, it must be the case that $ T $ does not respect additivity and is not a linear map. 
\end{q}

\end{document}