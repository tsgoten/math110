\documentclass[10pt, twocolumn]{article}
\author{Tarang Srivastava}
\usepackage{amsmath, amssymb, amsthm, commath, chngcntr, enumitem, multirow, thmtools, xcolor}
\usepackage{graphicx}
\usepackage[margin=.25in]{geometry}
\setlength{\columnsep}{.5in}
\newcommand{\C}{\mathbb{C}}
\newcommand{\question}[1]{\textcolor{blue}{#1} \\}
\newcommand{\R}{\mathbb{R}}
\newcommand{\F}{\mathbb{F}}
\newcommand{\N}{\mathbb{N}}
\newcommand{\LinearMap}[2]{\mathcal{L}(#1, #2)}
\newcommand{\poly}[2]{\mathcal{P}_{#1}\left(#2\right)}
\newcommand{\vspan}[1]{\text{span}\left(#1\right)}
\newcommand{\inv}[1]{#1^{-1}}
\newcommand{\todo}[1]{\textcolor{red}{TODO: #1} \\}
\newcommand{\nul}{\text{null }}
\newcommand{\nullity}{\text{nullity }}
\newcommand{\range}{\text{range }}
\newcommand{\rank}{\text{rank }}
\newcommand{\makechaptertitle}[1]{
\begin{center}
	\begin{large}
		#1
	\end{large}
	\begin{small}
		\\Tarang Srivastava
	\end{small}
\end{center}
}
\declaretheoremstyle[
spaceabove=\topsep, spacebelow=\topsep,
headfont=\normalfont\bfseries,
notefont=\bfseries, notebraces={Problem }{},
bodyfont=\normalfont,
postheadspace=0.5em,
name={\ignorespaces},
numbered=no,
headpunct=:]
{mystyle}
\declaretheorem[style=mystyle]{q}

\begin{document}
	
\makechaptertitle{Math 110 Homework 4}

\section{Exercies 3.B}
\begin{q}[7]
    Let $ V $ have dimension 2, and $ W $ have some dimension greater than or equal to 2. 
    Let $ T $ be a non-injective non-zero linear map $ T: V \to W $. 
    We know that $ \nul T $ is a subspace of $ V $. 
    Also, $ \nul T \neq \{0\} $ since it is not injective, 
    thus $ \dim \nul T > 0 $. 
    Since, we restricted $ V $ to have dimension 2 and $ \nul T $ is nonzero it must have exactly dimension 1. 
    There exists a subspace $ U $ of $ V $ such that $ V = \nul T \oplus U $. 
    Therefore, $ \dim U = 1 $.
    Let $ S $ be another  non-zero non-injective linear map $ S : V \to W $ such that $ \nul S = U $. 
    Now we have the case that for $ v \in V $, 
    $$ (S + T)(v) = Sv + Tv $$
    But for any non-zero $ v \in V $ it is either in $ \nul S $ or $ \nul T $ but not both. 
    So, the $ \nul (S + T) = \{0\} $, which means that $ S + T $ is injective. 
    Therefore, it is not a subspace of $ \LinearMap{U}{V} $ since it is not closed under addition.
\end{q}

\begin{q}[15]
    Assume for contradiction that $ T $ is a valid linear map. 
    Given that $ T $ is a map $ T: \R^5 \to \R^2$, we must have that $ \rank T = 2 $ and $ \nullity T = 3 $. 
    By 
    $$ \dim \R^5 = 5 = \rank + \nullity = 2 + 3 $$
    We then observe that the null space defined as follows
    has dimension 2. 
    $$\left\{\left(x_{1}, x_{2}, x_{3}, x_{4}, x_{5}\right) \in \mathbf{F}^{5}: x_{1}=3 x_{2} \text { and } x_{3}=x_{4}=x_{5}\right\}$$
    We claim that the vectors $ (3, 1, 0, 0, 0), (0, 0, 1, 1, 1) $ span the null space. 
    Observe for some arbitrary coefficients $ a, b \in \F $ the linear combination of the vectors 
    $ a(3, 1, 0, 0, 0) + b(0, 0, 1, 1, 1) = (3a, a, b, b, b)$. 
    This holds the required property that $ 3x_2 = x_1 $ and that $ x_3 = x_4 = x_5 $. 
    Thus, it spans our null space and we have a contradiction, since we found a spanning list that has length less than a linearly independent list.  
\end{q}

\begin{q}[19]
    We know that for a linear map $ T \in \LinearMap{V}{W} $ the dimension of the range is less than or equal to the codomain.
    That is, 
    $$ \rank T \leq \dim W $$
    Suppose, this was not the case then we could find a $ v \in V $ such that $ Tv \not \in W $ and that is not a linear map. 
    Then, we also have that $ \dim \nul T = \dim U $. 
    It then follows directly that 
    $$ \dim V = \rank T + \dim \nul T = \rank T + \dim U $$
    $$ \dim V - \rank T = \dim U $$
    Therefore, from our first inequality 
    $$ \dim U \geq \dim V - \dim W $$
\end{q}

\begin{q}[26]
    
\end{q}

\section{Exercises 3.C}
\begin{q}[2]
    For $ \poly{3}{\R} $ let the basis be the following basis $ 1, x, x^2, x^3 $ in this order. 
    And for $ \poly{2}{\R} $ let the basis be $ 3x^2, 2x, 1 $ for this order. 
    $ 1, x, x^2, x^3 $ is clearly a basis since it is just the standard basis reorganized, more formally the linear combination will still only be zero for all zero coefficients, since addition is commutative in $ \R $. 
    For  $ 3x^2, 2x, 1 $ any linear combination is only zero when all the coefficients are zero since each term has a different degree. 
\end{q}
\begin{q}[3]
    We want a matrix that basically has 1s in the diagonal up to the $ \rank T $ and 0s everywhere else. 
    So, we begin with the $ \nul T $. 
    We know that $ \nul T $ is a subspace of $ V $. 
    So let $ t_1, ..., t_m $ be a basis for $ \nul T $. 
    Then, since $ t_1, ..., t_m $ is a linearly independent list and is less than or equal to the length of the basis of $ V $ we can extend it to form a basis of $ V $.
    Note that $ \nullity T = m $
    Let $ v_1, ..., v_n $ be the addtional vectors we add to extend our list to be a basis of $ V $. 
    Note that $ \rank T = n $. 
    So, $ t_1, ..., t_m , v_1, ..., v_n $ is a basis of $ V $. 
    Because we collected all the vectors that are equal to zero we know that $ Tv_i = w_i $ for some $ w_i \in W $ for $ i \in \{1, ..., n\} $. 
    Now we can show these $ w_1, ..., w_n $ are linearly independent. 
    We begin with $ a_1w_1 + ... + a_n w_n = 0 $ for some coefficients $ a_i $. 
    We can substitute the $ Tv_i $ to get $ a_1Tv_1 + ... + a_n Tv_n = 0 $. 
    Collecting the terms we have $ T(a_1v_1 + ... + a_n v_n) = 0 $. 
    We know that $ v_1, ..., v_n $ are linearly independent since it is formed by dropping vectors from the basis of $ V $. 
    Then, $ a_1 v_1 + ... + a_n v_n = 0 $ only when all the coefficients are zero.
    So, we know all the $ a_i $ are zero, thus $ w_1, ..., w_n $ is linearly independent. 
    Then we can finally extend $ w_1, ..., w_n $ to form a basis for $ W $. 
    Now we basically our done since we have shown that we have a basis for which $ Tv_i = w_i $ for $ i \in \{1, ..., \rank T\} $. 
    So, we can make our desired matrix since everything else is supposed to be zero because it is in the null space. 
\end{q}
\begin{q}[4]
    In a very roundabout way this question is asking if there is a vector $ v $  in every basis of $ V $ such that for every 
    $ T \in \LinearMap{V}{W} $ it is that $ Tv = w $ for some $ w $ in the basis of $ W $. 
    The stright forward case is if for some $ v $ in the basis of $ V $ if $ Tv = 0 $ then put that $ v $ at the start of the basis list, so that the column can be all zeros and satisfy the condition.
    The second also straight forward case if if for some $ v $ in the basis of $ V $ we happen to have a basis for $ W $ such that $ Tv = w $ for $ w $ in the basis, then we are done as well, by moving both $ v $ and $ w $ to start of their respective basis lists.
    The last case is we can always construct a basis such that the condition is met, by just adding $ Tv = w $ to the basis of $ W $ and then removing all the vectors that are multiples of $ w $. 
    Then we will be left with a linearly independent list that we can always extend to form a basis of $ W $. 
\end{q}


\end{document}