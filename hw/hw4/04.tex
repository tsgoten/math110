\documentclass[10pt, twocolumn]{article}
\author{Tarang Srivastava}
\usepackage{amsmath, amssymb, amsthm, commath, chngcntr, enumitem, multirow, thmtools, xcolor}
\usepackage{graphicx}
\usepackage[margin=.25in]{geometry}
\setlength{\columnsep}{.5in}
\newcommand{\C}{\mathbb{C}}
\newcommand{\question}[1]{\textcolor{blue}{#1} \\}
\newcommand{\R}{\mathbb{R}}
\newcommand{\F}{\mathbb{F}}
\newcommand{\N}{\mathbb{N}}
\newcommand{\LinearMap}[2]{\mathcal{L}(#1, #2)}
\newcommand{\poly}[2]{\mathcal{P}_{#1}\left(#2\right)}
\newcommand{\vspan}[1]{\text{span}\left(#1\right)}
\newcommand{\inv}[1]{#1^{-1}}
\newcommand{\todo}[1]{\textcolor{red}{TODO: #1} \\}
\newcommand{\nul}{\text{null }}
\newcommand{\nullity}{\text{nullity }}
\newcommand{\range}{\text{range }}
\newcommand{\rank}{\text{rank }}
\newcommand{\makechaptertitle}[1]{
\begin{center}
	\begin{large}
		#1
	\end{large}
	\begin{small}
		\\Tarang Srivastava
	\end{small}
\end{center}
}
\declaretheoremstyle[
spaceabove=\topsep, spacebelow=\topsep,
headfont=\normalfont\bfseries,
notefont=\bfseries, notebraces={Problem }{},
bodyfont=\normalfont,
postheadspace=0.5em,
name={\ignorespaces},
numbered=no,
headpunct=:]
{mystyle}
\declaretheorem[style=mystyle]{q}

\begin{document}
	
\makechaptertitle{Math 110 Homework 4}

\section{Exercies 3.B}
\begin{q}[7]
    Given that $ T $ is injective. 
    For two unique vectors $ v, u \in V $. 
    
\end{q}

\begin{q}[15]
    Assume for contradiction that $ T $ is a valid linear map. 
    Given that $ T $ is a map $ T: \R^5 \to \R^2$, we must have that $ \rank T = 2 $ and $ \nullity T = 3 $. 
    By 
    $$ \dim \R^5 = 5 = \rank + \nullity = 2 + 3 $$
    We then observe that the null space defined as follows
    has dimension 2. 
    $$\left\{\left(x_{1}, x_{2}, x_{3}, x_{4}, x_{5}\right) \in \mathbf{F}^{5}: x_{1}=3 x_{2} \text { and } x_{3}=x_{4}=x_{5}\right\}$$
    We claim that the vectors $ (3, 1, 0, 0, 0), (0, 0, 1, 1, 1) $ span the null space. 
    Observe for some arbitrary coefficients $ a, b \in \F $ the linear combination of the vectors 
    $ a(3, 1, 0, 0, 0) + b(0, 0, 1, 1, 1) = (3a, a, b, b, b)$. 
    This holds the required property that $ 3x_2 = x_1 $ and that $ x_3 = x_4 = x_5 $. 
    Thus, it spans our null space and we have a contradiction, since we found a spanning list that has length less than a linearly independent list.  
\end{q}

\begin{q}[19]
    We know that for a linear map $ T \in \LinearMap{V}{W} $ the dimension of the range is less than or equal to the codomain.
    That is, 
    $$ \rank T \leq \dim W $$
    Suppose, this was not the case then we could find a $ v \in V $ such that $ Tv \not \in W $ and that is not a linear map. 
    Then, we also have that $ \dim \nul T = \dim U $. 
    It then follows directly that 
    $$ \dim V = \rank T + \dim \nul T = \rank T + \dim U $$
    $$ \dim V - \rank T = \dim U $$
    Therefore, from our first inequality 
    $$ \dim U \geq \dim V - \dim W $$
\end{q}

\begin{q}[26]
    
\end{q}

\end{document}