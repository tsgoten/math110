\documentclass[10pt, twocolumn]{article}
\author{Tarang Srivastava}
\usepackage{amsmath, amssymb, amsthm, chngcntr, enumitem, multirow, thmtools, xcolor}
\usepackage{graphicx}
\usepackage[margin=.25in]{geometry}
\setlength{\columnsep}{.5in}
\newcommand{\C}{\mathbb{C}}
\newcommand{\question}[1]{\textcolor{blue}{#1} \\}
\newcommand{\R}{\mathbb{R}}
\newcommand{\N}{\mathbb{N}}
\newcommand{\inv}[1]{#1^{-1}}
\newcommand{\todo}[1]{\textcolor{red}{TODO: #1} \\}
\newcommand{\makechaptertitle}[1]{
\begin{center}
	\begin{large}
		#1
	\end{large}
	\begin{small}
		\\Tarang Srivastava
	\end{small}
\end{center}
}
\declaretheoremstyle[
spaceabove=\topsep, spacebelow=\topsep,
headfont=\normalfont\bfseries,
notefont=\bfseries, notebraces={}{},
bodyfont=\normalfont,
postheadspace=0.5em,
name={\ignorespaces},
numbered=no,
headpunct=:]
{mystyle}
\declaretheorem[style=mystyle]{q}

\begin{document}
	
\makechaptertitle{Math 110 Homework 1}

\section*{Chapter 1.A}
\begin{q}[Problem 11]
    By the definition of multiplication of scalars and lists we know that $ \lambda $ must satisfy all the following equations.
    \begin{align*}
        \lambda (2 - 3i) &= 12 - 5i  \\
        \lambda (5+4i) &= 7 + 22i \\
        \lambda (-6 + 7i) &= -32 - 9i
        \intertext{Solving for the first two equations gives us that}
        \lambda = 3 + 2i
    \end{align*}
    But for the last equation we find that this value for $ \lambda $ does not work. 
        Therefore no such $ \lambda $ exists that satisfies all three equations and therefore the larger equation.
\end{q}

\section*{Chapter 1.B}
\begin{q}[Problem 1]
    By the existence of an additive inverse for all $ v $ we have
    \begin{align*}
        -(-v) + (-v) & = 0 \\
        \intertext{adding $ v $ to both sides}
        -(-v) + (-v) + v & = 0 + v \\
        \intertext{by associativity}
        -(-v) + ((-v) + v) & = v \\
        -(-v) + 0 & = v
    \end{align*}
    $ -(-v) = v $ as desired.
\end{q}
\begin{q}[Problem 6]
    No, $ \R \cup \{\infty\} \cup \{-\infty\} $ is not a vector space, because it does not follow all the properties of a vector space.
    Specifically, consider associativity. 
    It must be true that, 
    \begin{align*}
        ((-\infty) + (-\infty)) + \infty &= (-\infty) + ((-\infty) + \infty) \\
        0 & \neq (- \infty)
    \end{align*}
    Therefore associativity does not hold and it is not a vector space.
\end{q}

\section*{Chapter 1.C}
\begin{q}[Problem 1(c)]
    No. Consider the following counter example. 
    Let $ a = (1, 1, 0) $. This is in the subspace since $ 1 \cdot 1 \cdot 0 = 0 $ 
    and let  $ b = (1, 0, 1) $. This is in the subspace since $ 1 \cdot 0 \cdot 1 = 0 $ 
    For the subspace to be closed under addition $ a + b $ must be in the subspace, 
    but $ a + b = (1, 1, 1) $ which is not in the subspace since $ 1 \cdot 1 \cdot 1 = 1 $.
\end{q}
\begin{q}[Problem 1(d)]
    Yes. It is closed under vector addition and scalar multiplication. The additive identitiy can be shown from these two as noted in lecture.\\
    Closure under addition: \\
    Let $ x = (x_1, x_2, x_3) $ and $ y = (y_1, y_2, y_3) $ such that they are in the subspace.
    For $ x + y $ to be in the subspace, we need to show that $ x_1 + y_1 = 5(x_3 + y_3) $. 
    Which follows from the fact that $ x_1 + y_1 = 5x_3 + 5y_3 = 5(x_3 + y_3) $  by factoring out the 5. 
    Also note that $ x_1 + y_1 \in \mathbb{F} $ so that part is definitely in the subspace.
    Closure under multiplication: \\
    Let $ x = (x_1, x_2, x_3) $ and $ c \in \mathbb{F} $ it holds that $ c x_1 = 5c x_3 $ which is then just $ x_1 = 5 x_3 $ 
    therefore it is closed under scalar multiplication. Also, note that $ cx_1 \in \mathbb{F} $ so that condition is met as well for all the elements.
\end{q}
\begin{q}[Problem 3]
    Yes. The following is a subspace. \\
    Clousure under addition: \\
    Let $ f, g \in \R^{(4, 4)} $ such that they are differentiable and $ f'(-1) = 3f(2) $ and $ g(-1) = 3g(2) $. 
    Firstly, the sum of differentiable functions is differentiable.
    Adding the two we get $$ f'(-1) + g'(-1) = 3f(2) + 3g(2) $$ 
    We can factor out the 3 and then use the definition for function addition 
    $$ (f'+ g')(-1) = 3((f+g)(2)) $$ which then by the definition of scalar multiplication is 
    $$ (f'+ g')(-1) = (3(f+g))(2) $$ as desired. \\
    Closure under multiplication: \\
    A scalar multiple of a differentiable function is still differentiable. 
    Given
    $$ \lambda f'(-1) = 3 \lambda f(2) $$
    using the definition of scalar multiplication we have
    $$ (\lambda f')(-1) = (\lambda 3 f)(2) $$
    as desired.
\end{q}
\begin{q}[Problem 5]
    No. The complex vector space is a vector space over $ \C $ so it is possible to multiply for example with a scalar $ i $, 
    in which case $ \R^2 $ is not closed under multiplication.
\end{q}
\begin{q}[Problem 7]
    Let $ U = \{(x, y) \in \R^2 : x^2 + y^2 = 1\} $. That is $ U $ the unit circle. 
    We can always find an additive inverse, namely, $(-x, -y)$ for some $ x, y \in U $. 
    It is not a subspace because it does not have the additive identitiy.
\end{q}
\begin{q}[Problem 8]
    From question 1 part c. 
    Let $U =\{(x, y) \in \R^2 : xy = 0\} $
    Clearly, and $ c \in \R $ times the $ u \in U $ will still be in the subspace. Since, if $ xy = 0 \implies cxy = 0 $. 
    It is not closed under addtion as shown in Problem 1(c).
\end{q}
\begin{q}[Problem 9]
    Yes, it is a subspace because it is closed under vecotr addition and scalar multiplicaition.
    To show it is closed under addition let $f, g $ be periodic functions with periods $ p, q $ respectively in $ \R^\R $.
    Then, we must show that $ (f + g)(x) = (f+g)(x + r) $ for some $ r \in \R $. 
    Observe, we can select $ r = pq $. 
    For period functions we have the case that 
    $ f(x) = f(x+p) + f(x+pq) $;
    the same can be stated for $ g $. 
    Therefore, by defintion of addition of functions
    $$ (f + g)(x) = f(x) + g(x) = f(x + r) + g(x + r) = (f+g)(x + r) $$
    which shows that the sum of periodic functions are periodic as well over some period. 
    Showing closure under multiplicaition is more straightforward, since for some $ c \in \mathbb{F} $
    $ (cf)(x) = cf(x) = cf(x+p) = (cf)(x+p) $,
    which shows that after scalar multiplication it is still a periodic function. 
\end{q}
\begin{q}[Problem 10]
    Yes, $ U_1 \cap U_2 $ form a subspace. 
    To show closure under addition let $ u_1, u_2 \in U_1 \cap U_2 $. 
    We will argue that $ u_1 + u_2 $ is in both $ U_1 $ and $ U_2 $. 
    Since, $ u_1 $ and $ u_2 $ are in $ U_1 $ and since $ U_1 $ is closed under addtion $ u_1 + u_2 \in U_1 $. 
    The exact same argument follows for $ U_2 $,
    Since, $ u_1 $ and $ u_2 $ are in $ U_2 $ and since $ U_2 $ is closed under addtion $ u_1 + u_2 \in U_2 $. 
    Therefore, $ u_1 + u_2 $ is in both $ U_1 \cap U_2 $ and is closed under addition.  \\
    To show closure under scalar multiplication let $ \lambda \in \mathbb{F} $ and $ u \in U_1 \cap U_2 $. 
    Since, $ U_1 $ is closed under scalar multiplication $ \lambda u \in U_1 $.
    Similarily,  
    since, $ U_2 $ is closed under scalar multiplication $ \lambda u \in U_2 $.
    Therefore, $ \lambda u \in U_1 \cap U_2 $.
    Thus, we have shown that $ U_1 \cap U_2 $ forms a subspace. 
\end{q}
\begin{q}[Problem 15]
    From the defintion of addition of vector spaces 
    $$ U + U = \{u_1 + u_2 : u_1, u_2 \in U\}  $$
    Thus, $ U+ U $ is the sum of every pair of vectors in $ U $. 
    $ U + U $ is a subspace of $ V $. Since, it is closed under addition because it is just addition of four terms in $ U $ which is closed under addition. 
    It is also closed under addition because for some $ w \in U + U $ we have that $ w = u_1 + u_2 $ then $ \lambda w = \lambda u_1 + \lambda u_2 $ which all exist in $ U $ since $ U $ is closed under scalar multiplicaition.
\end{q}
\begin{q}[Problem 23]
    No, the following statement is not true. 
    Consider the following counterexample. 
    Let $ V = \R^2 $, which is a known vector space from the textbook.  
    Let $ W = {(x, x): x \in \R }. $ which is a vector space. 
    To show that $ W $ is closed under addition let $ w_1 = (a, a), w_2 = (b, b) \in W $. 
    Then $ w_1 + w_2 = (a+b, a+b) \in W $. 
    To show $ W $ is closed under multiplication let some constant $ c \in \R $ multiply some vector $ w = (a, a) \in W $. 
    Then, $ cw = (ca, ca) \in W $ since $ ca \in \R $ and is closed under scalar multiplicaition. 
    Thus, $ W $ is a subspace. 
    Now, let $ U_1 = {(x, 0) : x \in \R } $ and $ U_2 = {(0, x) : x \in \R } $ which are both subspaces from examples in the textbook.
    We, can show that $ U_1 \cap W = {(0, 0)} $, since it is the only value for $ x $ where all elements in the vector are the same. 
    By the same argument $ U_2 \cap W = {(0, 0)} $. 
    So, the shows that both are valid direct sums. 
    We know that $ U_1 \oplus W = (x+y, x) $ and $ U_2 \oplus W = (x, x+y) $. 
    It is now clear that we can express any arbitrary $ (a, b) $ as $ (x, x+y) $ since we can find a $ y $ such that $ y = b - a $. 
    Therefore, they both are the same direct sum and they both equal $ V $, but by construction $ U_1 \neq U_2 $. 
\end{q}

\section*{2.A}
\begin{q}[Problem 3]
    $ t = 2 $. 
    Observe
    $$ -3 \cdot (3, 1, 4) + 2 \cdot (2, -3, 5) + 1 \cdot (5, 9, \textbf{2}) = (0, 0 , 0) $$]
    Therefore, for $ t = 2$ it is not linearly independent in $ \R ^3 $.
\end{q}
\begin{q}[Problem 5(a)]
    For $ a, b \in \R $ we have 
    $$ a \cdot ( 1+ i) + b \cdot (1 - i) = (a + b) + (a-b) i = 0 $$
    which is only zero for when $ a = 0 $ and $ b = 0 $ therefore it is linearly independent.
\end{q}
\begin{q}[Problem 5(b)]
    Observe
    $ i \cdot (1 + i)  + (-i) \cdot (1 -i) = 0 $
    therefore it is not linearly independent, 
    because we can choose nonzero coeficients such that the linear combination is equal to zero.
\end{q}
\begin{q}[Problem 6]
    Consider the arbitrary linear combination 
    $$ a(v_1 - v_2) + b(v_2 - v_3) + c(v_3 - v_4) + d v_4 $$
    which rearraning gives us 
    $$ av_1 + (b-a)v_2 + (c-b)v_3 + (d-c) v_4 $$
    which for us to have a linear combination equal zero we must select $ a = 0$. 
    Which implies that $ b = 0$ which then implies $ c = 0 $ and that implies $ d = 0 $. 
    Therefore, the only combination that results in the linear combination being zero is for all the coefficients to be zero.
    Therefore, it is linearly independent. 
    Also, note that since we know the vectors are linearly independent they are not a scalar multiplicaitve of each other, therefore we can require that $ a = 0 $ to start the chain of implicaitons. 
\end{q}



\end{document}