\documentclass[10pt, twocolumn]{article}
\author{Tarang Srivastava}
\usepackage{amsmath, amssymb, amsthm, chngcntr, enumitem, multirow, xcolor}
\usepackage{graphicx}
\usepackage[margin=.25in]{geometry}
\setlength{\columnsep}{.5in}
\newcommand{\C}{\mathbb{C}}
\newcommand{\question}[1]{\textcolor{blue}{#1} \\}
\newcommand{\R}{\mathbb{R}}
\newcommand{\N}{\mathbb{N}}
\newcommand{\inv}[1]{#1^{-1}}
\newcommand{\todo}[1]{\textcolor{red}{TODO: #1} \\}
\newcommand{\makechaptertitle}[1]{
\begin{center}
	\begin{large}
		#1
	\end{large}
	\begin{small}
		\\Tarang Srivastava
	\end{small}
\end{center}
}
\theoremstyle{definition} 
\newtheorem{q}{}
\renewcommand*{\theq}{\alph{q}}
\counterwithin*{q}{section}
\begin{document}
	
\makechaptertitle{Math 110 Homework 1}

\section{Chapter 1.A}
\begin{q}
    Problem 11 \\
    \question{
        Explain why there does not exits $ \lambda \in \C $ such that
        $$ \lambda( 2- 3i, 5+4i, -6 + 7i) = (12 -5i, 7 + 22i, -32 - 9i) $$
    }
    By the definition of multiplication of scalars and lists we know that $ \lambda $ must satisfy all the following equations.
    \begin{align*}
        \lambda (2 - 3i) &= 12 - 5i  \\
        \lambda (5+4i) &= 7 + 22i \\
        \lambda (-6 + 7i) &= -32 - 9i
        \intertext{Solving for the first two equations gives us that}
        \lambda = 3 + 2i
        \intertext{But for the last equation we find that this value for $ \lambda $ does not work. 
        Therefore no such $ \lambda $ exists that satisfies all three equations and therefore the larger equation.}
    \end{align*}
\end{q}

\section{Chapter 1.B}
\begin{q}
    Problem 1 \\
    \question{Prove that $ -(-v) = v $ for every $ v \in V $}
    By the existence of an additive inverse for all $ v $ we have
    \begin{align*}
        -(-v) + (-v) & = 0 \\
        \intertext{adding $ v $ to both sides}
        -(-v) + (-v) + v & = 0 + v \\
        \intertext{by associativity}
        -(-v) + ((-v) + v) & = v \\
        -(-v) + 0 & = v \\
        \intertext{ $ -(-v) = v $ as desired.}
    \end{align*}
\end{q}
\begin{q}
    Problem 6 \\ 
    \question{page 17 Axler}
    No, $ \R \cup \{\infty\} \cup \{-\infty\} $ is not a vector space, because it does not follow all the properties of a vector space.
    Specifically, consider associativity. 
    It must be true that, 
    \begin{align*}
        ((-\infty) + (-\infty)) + \infty &= (-\infty) + ((-\infty) + \infty) \\
        0 & \neq (- \infty)
    \end{align*}
    Therefore associativity does not hold and it is not a vector space.
\end{q}

\section{Chapter 1.C}
\begin{q}
    Problem 1(c) \\
    \question{Determine whether it is a subspace}
    No. Consider the following counter example. 
    Let $ a = (1, 1, 0) $. This is in the subspace since $ 1 \cdot 1 \cdot 0 = 0 $ 
    and let  $ b = (1, 0, 1) $. This is in the subspace since $ 1 \cdot 0 \cdot 1 = 0 $ 
    For the subspace to be closed under addition $ a + b $ must be in the subspace, 
    but $ a + b = (1, 1, 1) $ which is not in the subspace since $ 1 \cdot 1 \cdot 1 = 1 $.
\end{q}
\begin{q}
    Problem 1(d) \\
    \question{Determine whether it is a subspace}
    Yes. It is closed under vector addition and scalar multiplication. The additive identitiy can be shown from these two as noted in lecture.\\
    Closure under addition: \\
    Let $ x = (x_1, x_2, x_3) $ and $ y = (y_1, y_2, y_3) $ such that they are in the subspace.
    For $ x + y $ to be in the subspace, we need to show that $ x_1 + y_1 = 5(x_3 + y_3) $. 
    Which follows from the fact that $ x_1 + y_1 = 5x_3 + 5y_3 = 5(x_3 + y_3) $  by factoring out the 5. 
    Also note that $ x_1 + y_1 \in \mathbb{F} $ so that part is definitely in the subspace.
    Closure under multiplication: \\
    Let $ x = (x_1, x_2, x_3) $ and $ c \in \mathbb{F} $ it holds that $ c x_1 = 5c x_3 $ which is then just $ x_1 = 5 x_3 $ 
    therefore it is closed under scalar multiplication. Also, note that $ cx_1 \in \mathbb{F} $ so that condition is met as well for all the elements.
\end{q}
\begin{q}
    Problem 3 \\
\end{q}
\begin{q}
    Problem 5 \\
    No. The complex vector space is a vector space over $ \C $ so it is possible to multiply for example with a scalar $ i $, 
    in which case $ \R^2 $ is not closed under multiplication.
\end{q}
\begin{q}
    Problem 7 \\
    Let $ U = \{(x, y) \in \R^2 : x^2 + y^2 = 1\} $. That is $ U $ the unit circle. 
    We can always find an additive inverse, namely, $(-x, -y)$ for some $ x, y \in U $. 
    It is not a subspace because it does not have the additive identitiy.
\end{q}
\begin{q}
    Problem 8 \\
    From question 1 part c. 
    Let $U =\{(x, y) \in \R^2 : xy = 0\} $
    Clearly, and $ c \in \R $ times the $ u \in U $ will still be in the subspace. Since, if $ xy = 0 \implies cxy = 0 $. 
    It is not closed under addtion as shown in Problem 1(c).
\end{q}



\end{document}