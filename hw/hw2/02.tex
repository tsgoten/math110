\documentclass[10pt, twocolumn]{article}
\author{Tarang Srivastava}
\usepackage{amsmath, amssymb, amsthm, chngcntr, enumitem, multirow, thmtools, xcolor}
\usepackage{graphicx}
\usepackage[margin=.25in]{geometry}
\setlength{\columnsep}{.5in}
\newcommand{\C}{\mathbb{C}}
\newcommand{\question}[1]{\textcolor{blue}{#1} \\}
\newcommand{\R}{\mathbb{R}}
\newcommand{\F}{\mathbb{F}}
\newcommand{\N}{\mathbb{N}}
\newcommand{\vspan}{\text{span}}
\newcommand{\inv}[1]{#1^{-1}}
\newcommand{\todo}[1]{\textcolor{red}{TODO: #1} \\}
\newcommand{\makechaptertitle}[1]{
\begin{center}
	\begin{large}
		#1
	\end{large}
	\begin{small}
		\\Tarang Srivastava
	\end{small}
\end{center}
}
\declaretheoremstyle[
spaceabove=\topsep, spacebelow=\topsep,
headfont=\normalfont\bfseries,
notefont=\bfseries, notebraces={}{},
bodyfont=\normalfont,
postheadspace=0.5em,
name={\ignorespaces},
numbered=no,
headpunct=:]
{mystyle}
\declaretheorem[style=mystyle]{q}

\begin{document}
	
\makechaptertitle{Math 110 Homework 2}

\section{Exercise 2.A}
\begin{q}[Problem 8]
    The statement holds. 
    Assume for contradiction that $ \lambda v_1, .., \lambda v_m $ is linearly dependent. 
    That is, for $ \lambda \neq 0 $, $ \lambda v_1 +  ... + \lambda v_m = 0 $. 
    Then we have a contradiction, since we claimed that $ v_1, ..., v_m $ are linearly independent,
    and we have an example of a linear combination with non-zero coefficients that is equal to zero. 
    Then, $ \lambda v_1, .., \lambda v_m $ must be linearly independent.
\end{q}
\begin{q}[Problem 9]
    The statement is false. 
    Consider the counterexample where $ w_i = -v_i $ for $ i \in \{1, ..., m\} $. 
    We know that $ w_i \in V $, since the additive inverse must exist in $ V $ for $ v_i $. 
    Then, for non-zero coefficients the linear combination $ (v_1 + w_1) +  ... + (v_m + w_m) = 0 $ 
    which shows that it is linearly dependent.
\end{q}
\begin{q}[Problem 10]
    To show that $ w \in \vspan(v_1, ..., v_m) $ we just have to show there exists a linear combination 
    such that $$ w = a_1v_1 + ...+ a_m v_m $$ 
    By linear dependence we have $ a_1(v_1 + w) + ... + a_1(v_m + w) = 0 $ for nonzero coefficients. 
    We can collect that $ w $ terms and bring it to the other side of the equal sign and divide by the coefficients, again because they are not all zero, to get
    $$ a_1 v_1 + ... + a_m v_m = (a_1 + ... + a_m) w $$ 
    $$ \frac{a_1}{(a_1 + ... + a_m)} v_1 + ... + \frac{a_m}{(a_1 + ... + a_m)} v_m = w $$
    Thus, showing that $ w $ is in the span of $ v_1, ..., v_m $. 
\end{q}
\begin{q}[Problem 11]
      
\end{q}

\end{document}