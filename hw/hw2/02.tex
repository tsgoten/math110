\documentclass[10pt, twocolumn]{article}
\author{Tarang Srivastava}
\usepackage{amsmath, amssymb, amsthm, chngcntr, enumitem, multirow, thmtools, xcolor}
\usepackage{graphicx}
\usepackage[margin=.25in]{geometry}
\setlength{\columnsep}{.5in}
\newcommand{\C}{\mathbb{C}}
\newcommand{\question}[1]{\textcolor{blue}{#1} \\}
\newcommand{\R}{\mathbb{R}}
\newcommand{\F}{\mathbb{F}}
\newcommand{\N}{\mathbb{N}}
\newcommand{\vspan}{\text{span}}
\newcommand{\inv}[1]{#1^{-1}}
\newcommand{\todo}[1]{\textcolor{red}{TODO: #1} \\}
\newcommand{\makechaptertitle}[1]{
\begin{center}
	\begin{large}
		#1
	\end{large}
	\begin{small}
		\\Tarang Srivastava
	\end{small}
\end{center}
}
\declaretheoremstyle[
spaceabove=\topsep, spacebelow=\topsep,
headfont=\normalfont\bfseries,
notefont=\bfseries, notebraces={}{},
bodyfont=\normalfont,
postheadspace=0.5em,
name={\ignorespaces},
numbered=no,
headpunct=:]
{mystyle}
\declaretheorem[style=mystyle]{q}

\begin{document}
	
\makechaptertitle{Math 110 Homework 2}

\section{Exercise 2.A}
\begin{q}[Problem 8]
    The statement holds. 
    Assume for contradiction that $ \lambda v_1, .., \lambda v_m $ is linearly dependent. 
    That is, for $ \lambda \neq 0 $, $ \lambda v_1 +  ... + \lambda v_m = 0 $. 
    Then we have a contradiction, since we claimed that $ v_1, ..., v_m $ are linearly independent,
    and we have an example of a linear combination with non-zero coefficients that is equal to zero. 
    Then, $ \lambda v_1, .., \lambda v_m $ must be linearly independent.
\end{q}
\begin{q}[Problem 9]
    The statement is false. 
    Consider the counterexample where $ w_i = -v_i $ for $ i \in \{1, ..., m\} $. 
    We know that $ w_i \in V $, since the additive inverse must exist in $ V $ for $ v_i $. 
    Then, for non-zero coefficients the linear combination $ (v_1 + w_1) +  ... + (v_m + w_m) = 0 $ 
    which shows that it is linearly dependent.
\end{q}
\begin{q}[Problem 10]
    To show that $ w \in \vspan(v_1, ..., v_m) $ we just have to show there exists a linear combination 
    such that $$ w = a_1v_1 + ...+ a_m v_m $$ 
    By linear dependence we have $ a_1(v_1 + w) + ... + a_1(v_m + w) = 0 $ for nonzero coefficients. 
    We can collect that $ w $ terms and bring it to the other side of the equal sign and divide by the coefficients, again because they are not all zero, to get
    $$ a_1 v_1 + ... + a_m v_m = (a_1 + ... + a_m) w $$ 
    $$ \frac{a_1}{(a_1 + ... + a_m)} v_1 + ... + \frac{a_m}{(a_1 + ... + a_m)} v_m = w $$
    Thus, showing that $ w $ is in the span of $ v_1, ..., v_m $. 
\end{q}
\begin{q}[Problem 11]
    To show the "$ \implies $" direction, assume for contradiction $ w \in \vspan(v_1, ..., v_m) $.
    Then, we can express $ w $ as a linear combination of $ v_1, ..., v_m $. 
    $ w = a_1 v_1 + ... + a_m v_m $
    Let $ a_1 v_1 + ... + a_m v_m - w $ be a linear combination, but here we have a contradiction. 
    Since, 
    $ a_1 v_1 + ... + a_m v_m - w = a_1 v_1 + ... + a_m v_m - (a_1 v_1 + ... + a_m v_m) = 0 $
    for non-zero coefficients $ a_1, ..., a_m $. 
    Therefore, $ w \not\in \vspan(v_1, ..., v_m) $. \\
    For the "$ \impliedby $" direction, assume for contradiction $ v_1, ..., v_m, w $ is linearly dependent. 
    Then, there exists a linear combination with nonzero coefficients such that
    $ a_1v_1 + ... + a_m v_m + cw = 0 $. 
    Subtracting by $ cw $ and dividing by $ c $, for when $ c \neq 0, $ if $ c = 0 $ we are done, 
    we get $ \frac{a_1}{c} v_1 + ... + \frac{a_m}{c} v_m = w $, which is a contradiction, since it implies that $ w \in \vspan(v_1, ..., v_m) $. 
    Thus, it must be that $ v_1, ..., v_m, w $ is linearly independent.
\end{q}

\section{Exercise 2.B}
\begin{q}[Problem 1]
    The zero-Vector space, $ V = \{0\} $ is the only vector space with exactly one basis, and that is the empty set. 
    Suppose there is another vector space with a basis $ b $ with length $ > 0 $. 
    Then, let $ v \in b $. 
    By the existence of an additive identity we know that $ -v $ is also in the vector space, and thus we can have an equally valid basis with $ \{-v\} $, and thus having two bases.
    Therefore, it must be the case that for a vector space to have exactly one basis, the basis must be of length 0, which only the empty set satisfies. 
\end{q}
\begin{q}[Problem 3]
    (a) A possible basis for $ U $ is as follows
    $ (3,1, 0, 0, 0), (0, 0, 7, 1, 0), (0, 0, 0, 0, 1) $
    observe that for any linear combination it holds that $ x_1 = 3x_2 $ and $ x_3 = 7x_4 $. \\
    (b) We can extend the previous basis by adding the following two vectors to get a basis for $ \R ^5 $.
    $ (3,1, 0, 0, 0), (0, 0, 7, 1, 0), (0, 0, 0, 0, 1), (1, 0, 0, 0, 0), (0, 0, 1, 0, 0) $
    Notice, that by adding the following vectors we can express any arbirary $ x_2 = a(3, 1, 0, 0, 0) - 3a(1, 0, 0, 0, 0) $. 
    The same reasoning follows for $ x_4$. \\
    (c) From part (b) we can just form a subspace that handles the case to express arbitrary $ x_2 $ and $ x_4 $.
    Therefore, let $ W = \{(x, 0, y, 0, 0) \colon x, y \in \R \} $. That is, $ W = \vspan((1, 0, 0 0, 0), (0, 0, 1, 0, 0))$. 
    We then also observe that $ U \cap W = \{0\} $ as desired. 
\end{q}
\begin{q}[Problem 5]
    Consider the following basis 
    $$ 1, x, x^3 + x^2, x^3 $$ 
    Thus, for any arbitrary polynomial of degree two you can select $ a(x^3 + x^2) - a(x^3) $. 
\end{q}
\begin{q}[Problem 7]
    
\end{q}

\end{document}