\documentclass[10pt, twocolumn]{article}
\usepackage{../hw}
\title{Homework 12}
\author{Tarang Srivastava}

\begin{document}
\makechaptertitle

\section{Exercises 7.D}
\begin{q}[1]
    First we verify that the given operator
    $$ \sqrt{T^{*} T} v=\frac{\|x\|}{\|u\|}\langle v, u\rangle u $$
    is indeed positive. 
    Observe that when we calculate 
    $$ \ip{\sqrt{T^{*} T} v, v} = \frac{\|x\|}{\|u\|}|\langle v, u\rangle|^2 $$
    it is always a non-negative value, thus the operator is positive.
    Second we find the adjoint of $ T $. 
    Define $ T^* $ as follows
    $$ T^*v = \ip{v, x} u $$
    with the same fixed $ x $ and $ u $ as for $ T $.
    This is a valid adjoint since for $ v, w \in V $ we have
    \begin{align*}
        \ip{Tv, w} &= \ip{v, T^*w} \\
        \ip{v, u}\ip{x, w} &= \ip{v, u}\overline{\ip{w, x}} = \ip{v, u}\ip{x, w}
    \end{align*}
    For all $ v \in V $ we have that 
    \begin{align*}
        T^* T v &= T^*(\ip{v, u} x) \\
        &= \ip{v, u} \ip{x, x} u
    \end{align*}
    Then, we can verify that indeed $ \sqrt{T^* T}^2 = T^*T $ that we calculated.
    Therefore, given that it is positive it is a valid square root of $ T^*T $.
\end{q}

\begin{q}[2]
    Let $ T $ be the operator represented by this $ 2 \times 2 $ matrix with the standard basis.
    $$
    T =
        \left(
        \begin{array}{cc}
            -5/2  & 5/2 \\
            -5/2 & 5/2
        \end{array}
        \right)
    $$
    The eigenvalues for $ T $ are exactly $ \lambda = 0 $.
    Then we find $ T^* T $ using the adjoint of the matrix provided. 
    $$ 
    T^* T = 
        \left(
        \begin{array}{cc}
            25/2  & -25/2 \\
            -25/2 & 25/2
        \end{array}
        \right)
    $$
    The eigenvalues for which are equal to $ \lambda = 25, 0 $. 
    So the singular values are equal to $ \sigma = 5, 0 $.
\end{q}

\begin{q}[4]
    Consider the polar decomposition for $ T $  
    $$ T = S \sqrt{T^* T}$$
    Then let $ v \in V $ be such that it is an eigenvector for $ \sqrt{T^* T} $ with the associated eigenvalue $ s $.
    So, 
    $$ Tv = S(sv) $$
    $$ \norm{Tv} = \norm{S(sv)} = \abs{s}\norm{Sv} $$
    Given that $ S $ is an isometry we then have
    $$ \norm{Tv} = s \norm{v} = s $$
    Since, $ s $ is nonnegative and the norm of $ v $ is 1. 
    By polar decomposition there always exists $ \sqrt{T^* T} $ thus an eigenvector associated with it.
\end{q}

\begin{q}[10]
    The singular values are the eigenvalues of $ \sqrt{\adjoint{T}T} $. 
    If $ T $ is self adjoint then $ \sqrt{\adjoint{T}T} = \sqrt{T^2} $. 
    From previous exercises we know that the eigenvalues for $ T^2 $ are just the eigenvalues $ \lambda $ for $ T $ squared, that is $ \lambda^2 $. 
    So we have that for all eigenvectors $ v $ of $ \sqrt{\adjoint{T}T} $
    $ \sqrt{T^2}v = sv $ such that $ T^2v = s^2v = \lambda^2 v $. 
    Taking the square root on both sides we get that $ s = \abs{s} = \abs{\lambda} $.
\end{q}

\begin{q}[11]
    First observe that the singular values for $ T^* $ are the eigenvalues of 
    $ \sqrt{TT^*} $.
    We know that $ T $ and $ \adjoint{T} $ have the same eigenvectors,
    and the associated eigenvalues are $ \lambda $ and $ \overline{\lambda} $ respectively. 
    Then for some eigenvector we have 
    $$ T^*Tv = T^*\lambda v = \abs{\lambda}^2 v $$
    also
    $$ TT^*v = T^*\overline{\lambda} v = \abs{\lambda}^2 v $$
    Since, $ TT^* $ and $ \adjoint{T}T $ have the same positive eigenvalues. 
    Their, singular values are the same, equal to the positive square root of those same eigenvalues.
\end{q}

\begin{q}[12]
    Consider the vector space $ V = \F^2 $ and the operator $ T $ represented by the following matrix in the standard basis.
    $$
    T = 
        \left(
            \begin{array}{cc}
                0 & 0 \\
                -1 & 0 \\ 
            \end{array}
        \right)
    $$
    Then this operator has the singular values $ \sigma = 1, 0 $.
    Observe that, $ T^2 = 0 $, so it has the singular values $ \sigma = 0 $ 
    So, the singular values aren't even equal. 
\end{q}

\begin{q}[13]
    $ \implies $
    First note that $ \sqrt{0} = 0 = 0^2 $.
    If $ T $ is invertible, then $$ \nul \adjoint{T} = (\range T)^\perp = V^\perp = \emptyspace $$
    So, $ T^* $ is invertible, and then $ T^*T $ is invertible.  
    Then from a pervious exercise, $ T^* T $ does not have zero as an eigenvalue.
    Since the singular values are the positive square root of the eigenvalues of $ \adjoint{T} T $ and it has non-zero eigenvalues, 
    then the singular values are all non-zero as well. \\
    $ \impliedby $ Consider the polar decomposition of $ T $. 
    If all the singular values are nonzero, then $ \sqrt{\adjoint{T} T} $ has all nonzero eigenvalues.
    Then, $ \nul \sqrt{\adjoint{T} T} = \emptyspace $ so it is inveritble. 
    Since, $ S $ is an isometry and we have a composition of invertible operators $ S $ and $ \sqrt{\adjoint{T} T} $ which is invertible, 
    Therefore, $ T $ is invertible.
\end{q}

\section{8.A}

\begin{q}[3]
     
\end{q}

\begin{q}[4]
    Assume for contradiction that the intersection is not empty, 
    $$ G(\alpha, T) \cap G(\beta, T) \neq \emptyspace $$
    Let $ v $ be the eigenvector in the intersection
    $$ v \in G(\alpha, T) \cap G(\beta, T) $$
    Then construct the linearly independent list of eigenvectors as described in 8.13, choosing $ v $ to represent both $ \alpha $ and $ \beta $. 
    It is trivial now that that the list is not linearly independent, thus a contradiction.
\end{q}

\begin{q}[6]
    
\end{q}

\begin{q}[8]
    
\end{q}

\begin{q}[9]
    
\end{q}

\begin{q}[11]
    
\end{q}

\begin{q}[14]
    
\end{q}

\end{document}