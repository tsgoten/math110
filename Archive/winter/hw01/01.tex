\documentclass[10pt, twocolumn]{article}
\author{Tarang Srivastava}
\usepackage{amsmath, amssymb, amsthm, chngcntr, enumitem, multirow, xcolor}
\usepackage{graphicx}
\usepackage[margin=.25in]{geometry}
\setlength{\columnsep}{.5in}
\newcommand{\C}{\mathbb{C}}
\newcommand{\makechaptertitle}[1]{
\begin{center}
	\begin{large}
		#1
	\end{large}
	\begin{small}
		\\Tarang Srivastava
	\end{small}
\end{center}
}
\theoremstyle{definition} 
\newtheorem{q}{}
\renewcommand*{\theq}{\alph{q}}
\counterwithin*{q}{section}
\pagecolor{darkgray}
\begin{document}
\color{white}
	
\makechaptertitle{Math 110 Homework 1}

\section{Chapter 1.A}
\begin{q}
    Problem 11 \\
    Explain why there does not exits $ \lambda \in \C $ such that
    $$ \lambda( 2- 3i, 5+4i, -6 + 7i) = (12 -5i, 7 + 22i, -32 - 9i) $$
    By the definition of multiplication of scalars and lists we know that $ \lambda $ must satisfy all the following equations.
    \begin{align*}
        \lambda (2 - 3i) &= 12 - 5i  \\
        \lambda (5+4i) &= 7 + 22i \\
        \lambda (-6 + 7i) &= -32 - 9i
        \intertext{Solving for the first two equations gives us that}
        \lambda = 3 + 2i
        \intertext{But for the last equation we find that this value for $ \lambda $ does not work. 
        Therefore no such $ \lambda $ exists that satisfies all three equations and therefore the larger equation.}
    \end{align*}
\end{q}

\end{document}