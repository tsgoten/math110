% Math 110 Midterm 1 Cheatsheet
% Tarang Srivastava
\documentclass{article}
\author{Tarang Srivastava}
\usepackage[margin=.05in, landscape]{geometry}
\setlength{\columnsep}{.05in}
\usepackage{amsmath, amssymb, amsthm, commath, chngcntr, enumitem, multicol, multirow, thmtools, xcolor}
\usepackage{graphicx, wrapfig, float}
\usepackage[utf8]{inputenc}
\usepackage[T1]{fontenc}
\usepackage{ebgaramond}
\usepackage{soul}
% Settings
\setlength{\parindent}{0pt}
% Symbols
\newcommand{\ddd}{$\bullet$}
% Colors
\newcommand{\red}[1]{\textcolor{red}{#1}}
\newcommand{\green}[1]{\textcolor{green}{#1}}
\newcommand{\blue}[1]{\textcolor{blue}{#1}}
\newcommand{\pink}[1]{\textcolor{magenta}{#1}}
\newcommand{\orange}[1]{\textcolor{orange}{#1}}
\newcommand{\yellow}[1]{\textcolor{yellow}{#1}}
% Math Shortcuts
\newcommand{\C}{\mathbb{C}}
\newcommand{\question}[1]{\textcolor{blue}{#1} \\}
\newcommand{\R}{\mathbb{R}}
\newcommand{\F}{\mathbb{F}}
\newcommand{\N}{\mathbb{N}}
\newcommand{\poly}[2]{\mathcal{P}_{#1}\left(#2\right)}
\newcommand{\vspan}[1]{\text{span}\left(#1\right)}
\newcommand{\inv}[1]{#1^{-1}}
\newcommand{\todo}[1]{\textcolor{red}{TODO: #1} \\}
\newcommand{\nul}{\text{null }}
\newcommand{\range}{\text{range }}
% Headings
\newcommand{\mysection}[1]{\colorbox{yellow}{{\textbf{\textbf{\textit{\red{#1}}}}}}}
\newcommand{\defi}[1]{{\textbf{{\textit{\orange{#1:}}}}}}
\newcommand{\thrm}[1]{{\textbf{{\textit{\blue{#1:}}}}}}
\newcommand{\mysubsubsub}[1]{{{\blue{#1}}}}
\newcommand{\vocab}[1]{{\pink{#1}}}
% Pictures
\newcommand{\fig}[1]{
	\includegraphics[width=\columnwidth]{#1}
}
\newcommand{\figwidth}[2]{
	%file, width
	\includegraphics[width=#2cm]{#1}
}
\newcommand{\figtwo}[2]{
	%file, width
	\includegraphics[width=.5 \linewidth]{#1}
	\includegraphics[width=.5 \linewidth]{#2}
}
\newcommand{\figwrap}[1]{
	%file, width, height, side
	\begin{wrapfigure}{r}{0.5\textwidth}
		\begin{center}
			\includegraphics[width=0.48\textwidth]{#1}
		\end{center}
	\end{wrapfigure}
}

\begin{document}
	% Uncomment the line below to modify the font size
	%  \small
	 \tiny
    \begin{multicols*}{4}
        \mysection{Section 1.A -- $ \R^n $ and $ \C^n $}

% \defi{Complex Numbers}
% A complex number is an ordered pair $(a, b),$ where $a, b \in \mathbf{R},$ but we will write this as $a+b i$ \\
% The set of all complex numbers is denoted by $\mathbf{C}:$
% $
% \mathbf{C}=\{a+b i: a, b \in \mathbf{R}\}
% $
% Addition and multiplication on $\mathbf{C}$ are defined by
% $
% (a+b i)+(c+d i) =(a+c)+(b+d) i
% $
% $
% (a+b i)(c+d i) =(a c-b d)+(a d+b c) i
% $
% here $a, b, c, d \in \mathbf{R}$ \\

% \thrm{Properties of complex arithmetic} \\
% commutativity
% \begin{align*}
% \alpha+\beta=\beta+\alpha \text { and } \alpha \beta=\beta \alpha \text { for all } \alpha, \beta \in \mathbf{C}
% \end{align*}
% associativity
% \begin{align*}
% (\alpha+\beta)+\lambda=\alpha+(\beta+\lambda) \text { and }(\alpha \beta) \lambda=\alpha(\beta \lambda) \text { for all } \alpha, \beta, \lambda \in \mathbf{C}
% \end{align*}
% identities
% \begin{align*}
% \lambda+0=\lambda \text { and } \lambda 1=\lambda \text { for all } \lambda \in \mathbf{C}
% \end{align*}
% additive inverse for every $\alpha \in \mathbf{C},$ there exists a unique $\beta \in \mathbf{C}$ such that $\alpha+\beta=0$ \\
% multiplicative inverse for every $\alpha \in \mathbf{C}$ with $\alpha \neq 0,$ there exists a unique $\beta \in \mathbf{C}$ such that $\alpha \beta=1$ \\
% distributive property
% \begin{align*}
% \lambda(\alpha+\beta)=\lambda \alpha+\lambda \beta \text { for all } \lambda, \alpha, \beta \in \mathbf{C}
% \end{align*}

\defi{$ - \alpha $, subtraction, $ 1/ \alpha $, division}
Let $\alpha, \beta \in \mathbf{C}$
$\cdot$ Let $-\alpha$ denote the additive inverse of $\alpha .$ Thus $-\alpha$ is the unique complex number such that
$
\alpha+(-\alpha)=0
$
$\cdot$ Subtraction on $\mathrm{C}$ is defined by
$
\beta-\alpha=\beta+(-\alpha)
$
$\cdot$ For $\alpha \neq 0,$ let $1 / \alpha$ denote the multiplicative inverse of $\alpha .$ Thus $1 / \alpha$ is the unique complex number such that
$
\alpha(1 / \alpha)=1
$
$\cdot$ Division on $\mathbf{C}$ is defined by
$
\beta / \alpha=\beta(1 / \alpha)
$

\defi{list, length}
Suppose $n$ is a nonnegative integer. A list of length $n$ is an ordered collection of $n$ elements (which might be numbers, other lists, or more abstract entities) separated by commas and surrounded by parentheses. A list of length $n$ looks like this:
$
\left(x_{1}, \ldots, x_{n}\right)
$
Two lists are equal if and only if they have the same length and the same elements in the same order.

\defi{$\F^n$}
$\mathbf{F}^{n}$ is the set of all lists of length $n$ of elements of $\mathbf{F}:$
$
\mathbf{F}^{n}=\left\{\left(x_{1}, \ldots, x_{n}\right): x_{j} \in \mathbf{F} \text { for } j=1, \ldots, n\right\}
$
For $\left(x_{1}, \ldots, x_{n}\right) \in \mathbf{F}^{n}$ and $j \in\{1, \ldots, n\},$ we say that $x_{j}$ is the $j^{\text {th }}$ coordinate of $\left(x_{1}, \dots, x_{n}\right)$

\defi{addition in $\F^n$}
Addition in $\mathbf{F}^{n}$ is defined by adding corresponding coordinates:
$
\left(x_{1}, \ldots, x_{n}\right)+\left(y_{1}, \ldots, y_{n}\right)=\left(x_{1}+y_{1}, \ldots, x_{n}+y_{n}\right)
$

\thrm{Commutativity of addition in $ \F^n $}
If $x, y \in \mathbf{F}^{n},$ then $x+y=y+x$

\defi{0}
Let 0 denote the list of length $n$ whose coordinates are all 0 :
$
0=(0, \ldots, 0)
$

\defi{additive inverse in $ \F^n $}
For $x \in \mathbf{F}^{n},$ the additive inverse of $x,$ denoted $-x,$ is the vector $-x \in \mathbf{F}^{n}$ such that $x+(-x)=0$
In other words, if $x=\left(x_{1}, \ldots, x_{n}\right),$ then $-x=\left(-x_{1}, \ldots,-x_{n}\right)$

\defi{scalar multiplication in $ \F^n $}
The product of a number $\lambda$ and a vector in $\mathbf{F}^{n}$ is computed by multiplying each coordinate of the vector by $\lambda$ :
$
\lambda\left(x_{1}, \ldots, x_{n}\right)=\left(\lambda x_{1}, \ldots, \lambda x_{n}\right)
$
here $\lambda \in \mathbf{F}$ and $\left(x_{1}, \ldots, x_{n}\right) \in \mathbf{F}^{n}$ \\
		\mysection{Section 1.B -- Definition of Vector Space}

\defi{addition, scalar multiplication}
$\cdot$ An addition on a set $V$ is a function that assigns an element $u+v \in V$
to each pair of elements $u, v \in V$
$\cdot$ A scalar multiplication on a set $V$ is a function that assigns an element $\lambda v \in V$ to each $\lambda \in \mathbf{F}$ and each $v \in V$

\defi{Vector Space}
A vector space is a set $V$ along with an addition on $V$ and a scalar multiplication on $V$ such that the following properties hold:
commutativity
$$
u+v=v+u \text { for all } u, v \in V
$$
associativity $(u+v)+w=u+(v+w)$ and $(a b) v=a(b v)$ for all $u, v, w \in V$
$$
\text { and all } a, b \in \mathbf{F}
$$
additive identity there exists an element $0 \in V$ such that $v+0=v$ for all $v \in V$
additive inverse
for every $v \in V,$ there exists $w \in V$ such that $v+w=0$
multiplicative identity
$1 v=v$ for all $v \in V$
distributive properties
$$
a(u+v)=a u+a v \text { and }(a+b) v=a v+b v \text { for all } a, b \in \mathbf{F} \text { and }
$$
all $u, v \in V$

\defi{vector, point}
Elements of a vector space are called vectors or points.

\defi{real vector space, complex vector space}
$\bullet$ A vector space over $\mathbf{R}$ is called a real vector space.
$\cdot$ A vector space over $\mathbf{C}$ is called a complex vector space.

\defi{$\F^s$}
$\cdot$ If $S$ is a set, then $\mathbf{F}^{S}$ denotes the set of functions from $S$ to $\mathbf{F}$
$\cdot$ For $f, g \in \mathbf{F}^{S},$ the sum $f+g \in \mathbf{F}^{S}$ is the function defined by
$$
(f+g)(x)=f(x)+g(x)
$$
for all $x \in S$
$\bullet$ For $\lambda \in \mathbf{F}$ and $f \in \mathbf{F}^{S},$ the product $\lambda f \in \mathbf{F}^{S}$ is the function defined by
$$
(\lambda f)(x)=\lambda f(x)
$$
for all $x \in S$

\thrm{Unique Additive Identity}
A vector space has a unique additive identity

\thrm{Unique additive inverse}
Every element in a vector space has a unique additive inverse.

\thrm{The number 0 times a vector}
$0 v=0$ for every $v \in V$

\thrm{A number times the vector 0}
$a 0=0$ for every $a \in \mathbf{F}$

\thrm{The number -1 times a vector}
$(-1) v=-v$ for every $v \in V$ \\
		\mysection{Section 1.C -- Subspaces} \\

\defi{Subspace}
A subset $U$ of $V$ is called a subspace of $V$ if $U$ is also a vector space (using the same addition and scalar multiplication as on $V$ ).

\thrm{Conditions for a subspace}
A subset $U$ of $V$ is a subspace of $V$ if and only if $U$ satisfies the following three conditions:
additive identity
$0 \in U$
closed under addition $u, w \in U$ implies $u+w \in U$
closed under scalar multiplication
$a \in \mathbf{F}$ and $u \in U$ implies $a u \in U$

\defi{sum of subsets}
Suppose $U_{1}, \ldots, U_{m}$ are subsets of $V .$ The sum of $U_{1}, \ldots, U_{m},$ denoted $U_{1}+\cdots+U_{m},$ is the set of all possible sums of elements of $U_{1}, \ldots, U_{m}$ More precisely,
$$
U_{1}+\cdots+U_{m}=\left\{u_{1}+\cdots+u_{m}: u_{1} \in U_{1}, \ldots, u_{m} \in U_{m}\right\}
$$

\thrm{Sum of subspaces is the smalles containing subspace}
Suppose $U_{1}, \ldots, U_{m}$ are subspaces of $V .$ Then $U_{1}+\cdots+U_{m}$ is the smallest subspace of $V$ containing $U_{1}, \ldots, U_{m}$

\defi{direct sum}
Suppose $U_{1}, \ldots, U_{m}$ are subspaces of $V$
$\cdot$ The sum $U_{1}+\cdots+U_{m}$ is called a direct sum if each element of $U_{1}+\cdots+U_{m}$ can be written in only one way as a sum $u_{1}+\cdots+u_{m},$ where each $u_{j}$ is in $U_{j}$
$\cdot$ If $U_{1}+\cdots+U_{m}$ is a direct sum, then $U_{1} \oplus \cdots \oplus U_{m}$ denotes $U_{1}+\cdots+U_{m},$ with the $\oplus$ notation serving as an indication that this is a direct sum.

\thrm{Condition for a direct sum}
Suppose $U$ and $W$ are subspaces of $V .$ Then $U+W$ is a direct sum if and only if $U \cap W=\{0\}$

\thrm{Direct sum of two subspaces}
Suppose $U$ and $W$ are subspaces of $V .$ Then $U+W$ is a direct sum if and only if $U \cap W=\{0\}$ \\
		\mysection{Section 2.A Span and Linear Independence}

\defi{Span}
The set of all linear combinations of a list of vectors $v_{1}, \ldots, v_{m}$ in $V$ is called the span of $v_{1}, \ldots, v_{m},$ denoted $\operatorname{span}\left(v_{1}, \ldots, v_{m}\right) .$ In other words,
$$
\operatorname{span}\left(v_{1}, \ldots, v_{m}\right)=\left\{a_{1} v_{1}+\cdots+a_{m} v_{m}: a_{1}, \ldots, a_{m} \in \mathbf{F}\right\}
$$
The span of the empty list ( ) is defined to be $\{0\} .$

\thrm{Span is the smallest containing subspace}
The span of a list of vectors in $V$ is the smallest subspace of $V$ containing all the vectors in the list.

\defi{spans}
If $\operatorname{span}\left(v_{1}, \ldots, v_{m}\right)$ equals $V,$ we say that $v_{1}, \ldots, v_{m}$ spans $V$

\defi{finite-dimensional vector space}
A vector space is called finite-dimensional if some list of vectors in it spans the space.

\defi{polynomial over a field F}
A function $p: \mathbf{F} \rightarrow \mathbf{F}$ is called a polynomial with coefficients in $\mathbf{F}$ if there exist $a_{0}, \ldots, a_{m} \in \mathbf{F}$ such that
$$
p(z)=a_{0}+a_{1} z+a_{2} z^{2}+\cdots+a_{m} z^{m}
$$
for all $z \in \mathbf{F}$. 
$\mathcal{P}(\mathbf{F})$ is the set of all polynomials with coefficients in $\mathbf{F}$.

\defi{degree of a polynomial}
$\cdot$ A polynomial $p \in \mathcal{P}(\mathbf{F})$ is said to have degree $m$ if there exist scalars $a_{0}, a_{1}, \ldots, a_{m} \in \mathbf{F}$ with $a_{m} \neq 0$ such that
$$
p(z)=a_{0}+a_{1} z+\cdots+a_{m} z^{m}
$$
for all $z \in \mathbf{F} .$ If $p$ has degree $m,$ we write deg $p=m$ \\
$\cdot$ The polynomial that is identically 0 is said to have degree $-\infty$.

\defi{$ \poly{m}{\F} $}
For $m$ a nonnegative integer, $\mathcal{P}_{m}(\mathbf{F})$ denotes the set of all polynomials with coefficients in $\mathbf{F}$ and degree at most $m .$

\defi{infinite-dimensional vector space}
A vector space is called infinite-dimensional if it is not finite-dimensional.

\defi{linearly independent}
$\cdot$ A list $v_{1}, \ldots, v_{m}$ of vectors in $V$ is called linearly independent if the only choice of $a_{1} \ldots ., a_{m} \in \mathbf{F}$ that makes $a_{1} v_{1}+\cdots+a_{m} v_{m}$
equal 0 is $a_{1}=\cdots=a_{m}=0$ \\
$\cdot$ The empty list ( ) is also declared to be linearly independent.

\defi{linearly dependent}
$\cdot$ A list of vectors in $V$ is called linearly dependent if it is not linearly independent. \\
$\cdot$ In other words, a list $v_{1}, \ldots, v_{m}$ of vectors in $V$ is linearly dependent if there exist $a_{1}, \ldots, a_{m} \in \mathbf{F},$ not all $0,$ such that $a_{1} v_{1}+\cdots+a_{m} v_{m}=0$

\thrm{Linear Dependence Lemma}
Suppose $v_{1}, \ldots, v_{m}$ is a linearly dependent list in $V .$ Then there exists $j \in\{1,2, \ldots, m\}$ such that the following hold:
(a) $\quad v_{j} \in \operatorname{span}\left(v_{1}, \ldots, v_{j-1}\right)$
(b) if the $j^{\text {th }}$ term is removed from $v_{1}, \ldots, v_{m},$ the span of the remaining list equals $\operatorname{span}\left(v_{1}, \ldots, v_{m}\right)$

\thrm{Length of linearly independent list $ \leq $ length of spanning list}
In a finite-dimensional vector space, the length of every linearly independent list of vectors is less than or equal to the length of every spanning list of vectors.

\thrm{Finite-dimensional subspaces}
Every subspace of a finite-dimensional vector space is finite dimensional. 
 \\
		\mysection{Section 2.B Bases}

\defi{basis}
A basis of $V$ is a list of vectors in $V$ that is linearly independent and spans $V$

\thrm{Criterion for basis}
A list $v_{1}, \ldots, v_{n}$ of vectors in $V$ is a basis of $V$ if and only if every $v \in V$ can be written uniquely in the form
$
\quad v=a_{1} v_{1}+\cdots+a_{n} v_{n}
$
where $a_{1}, \ldots, a_{n} \in \mathbf{F}$

\thrm{Spanning list contains a basis}
Every spanning list in a vector space can be reduced to a basis of the vector space.

\thrm{Linearly independent list extends to a basis}
Every linearly independent list of vectors in a finite-dimensional vector space can be extended to a basis of the vector space.

\thrm{Every subspace $ V $ is part of a direct sum equal to $ V $.}
Suppose $V$ is finite-dimensional and $U$ is a subspace of $V .$ Then there is a subspace $W$ of $V$ such that $V=U \oplus W$
 \\
		\mysection{Section 2.C Dimension}

\defi{dimension, $ \dim V $}
The dimension of a finite-dimensional vector space is the length of any basis of the vector space.
The dimension of $V$ (if $V$ is finite-dimensional) is denoted by $\operatorname{dim} V$. 

\thrm{Dimension of  subspace}
If $V$ is finite-dimensional and $U$ is a subspace of $V,$ then $\operatorname{dim} U \leq \operatorname{dim} V$

\thrm{Linearly independent list of the right length is a basis}
Suppose $V$ is finite-dimensional. Then every linearly independent list of vectors in $V$ with length $\operatorname{dim} V$ is a basis of $V$

\thrm{Spanning list of the right length is a basis}
Suppose $V$ is finite-dimensional. Then every spanning list of vectors in $V$ with length $\operatorname{dim} V$ is a basis of $V$

\thrm{Dimension of a sum}
If $U_{1}$ and $U_{2}$ are subspaces of a finite-dimensional vector space, then
$$
\operatorname{dim}\left(U_{1}+U_{2}\right)=\operatorname{dim} U_{1}+\operatorname{dim} U_{2}-\operatorname{dim}\left(U_{1} \cap U_{2}\right)
$$
 \\
		\mysection{Section 3.A The Vector Space of Linear Maps}

\defi{linear map}
A linear map from $V$ to $W$ is a function $T: V \rightarrow W$ with the following properties:
additivity
$$
T(u+v)=T u+T v \text { for all } u, v \in V
$$
homogeneity
$$
T(\lambda v)=\lambda(T v) \text { for all } \lambda \in \mathbf{F} \text { and all } v \in V
$$

\defi{Notation $ \mathcal{L}(V, W) $}
The set of all linear maps from $V$ to $W$ is denoted $\mathcal{L}(V, W)$

\thrm{Linear maps and basis of domain}
Suppose $v_{1}, \ldots, v_{n}$ is a basis of $V$ and $w_{1}, \ldots, w_{n} \in W .$ Then there exists a unique linear map $T: V \rightarrow W$ such that
$$
T v_{j}=w_{j}
$$
for each $j=1, \ldots, n$

\defi{additiona nd scalar multiplication on linear maps}
Suppose $S, T \in \mathcal{L}(V, W)$ and $\lambda \in \mathbf{F} .$ The sum $S+T$ and the product $\lambda T$ are the linear maps from $V$ to $W$ defined by
$$
(S+T)(v)=S v+T v \quad \text { and } \quad(\lambda T)(v)=\lambda(T v)
$$
for all $v \in V$

\thrm{$\mathcal{L}(V, W) $ is a vector space}
With the operations of addition and scalar multiplication as defined above, $\mathcal{L}(V, W)$ is a vector space.

\defi{Product of Linear Maps}
If $T \in \mathcal{L}(U, V)$ and $S \in \mathcal{L}(V, W),$ then the product $S T \in \mathcal{L}(U, W)$ is defined by
$$
(S T)(u)=S(T u)
$$
for $u \in U$

\thrm{Algebraic Properties of products of linear maps}
\textbf{associativity}
$$
\left(T_{1} T_{2}\right) T_{3}=T_{1}\left(T_{2} T_{3}\right)
$$
whenever $T_{1}, T_{2},$ and $T_{3}$ are linear maps such that the products make sense (meaning that $T_{3}$ maps into the domain of $T_{2},$ and $T_{2}$ maps into the domain of $T_{1}$ ).
\textbf{identity}
$$
T I=I T=T
$$
whenever $T \in \mathcal{L}(V, W)$ (the first $I$ is the identity map on $V,$ and the second $I$ is the identity map on $W$ ).
\textbf{distributive properties}
$$
\left(S_{1}+S_{2}\right) T=S_{1} T+S_{2} T \quad \text { and } \quad S\left(T_{1}+T_{2}\right)=S T_{1}+S T_{2}
$$
whenever $T, T_{1}, T_{2} \in \mathcal{L}(U, V)$ and $S, S_{1}, S_{2} \in \mathcal{L}(V, W)$

\thrm{Linear maps take 0 to 0}
Suppose $T$ is a linear map from $V$ to $W .$ Then $T(0)=0$
 \\
		\mysection{Section 3.B} \\
		\mysection{Section 3.C Matrices}

\defi{matrix, $ A_{j, k}$}
Let $m$ and $n$ denote positive integers. An $m$ -by-n matrix $A$ is a rectangular array of elements of $\mathbf{F}$ with $m$ rows and $n$ columns:
$$
A=\left(\begin{array}{ccc}
{A_{1,1}} & {\dots} & {A_{1, n}} \\
{\vdots} & {} & {\vdots} \\
{A_{m, 1}} & {\dots} & {A_{m, n}}
\end{array}\right)
$$
The notation $A_{j, k}$ denotes the entry in row $j,$ column $k$ of $A .$ In other words, the first index refers to the row number and the second index refers to the column number.

\defi{matrix of a linear map, $ \mathcal{M} (T)$}
Suppose $T \in \mathcal{L}(V, W)$ and $v_{1}, \ldots, v_{n}$ is a basis of $V$ and $w_{1}, \ldots, w_{m}$ is a basis of $W .$ The matrix of $T$ with respect to these bases is the $m-b y-n$ matrix $\mathcal{M}(T)$ whose entries $A_{j, k}$ are defined by
$$
T v_{k}=A_{1, k} w_{1}+\cdots+A_{m, k} w_{m}
$$
If the bases are not clear from the context, then the notation $\mathcal{M}\left(T,\left(v_{1}, \ldots, v_{n}\right),\left(w_{1}, \ldots, w_{m}\right)\right)$ is used.

\defi{$\F^{m, n}$}
For $m$ and $n$ positive integers, the set of all $m-b y-n$ matrices with entries in $\mathbf{F}$ is denoted by $\mathbf{F}^{m, n}$

\thrm{$\dim \F^{m, n} = mn$} 
Suppose $m$ and $n$ are positive integers. With addition and scalar multiplication defined as above, $\mathbf{F}^{m, n}$ is a vector space with dimension $m n$

\defi{matrix multiplication}
Suppose $A$ is an $m-$ by $-n$ matrix and $C$ is an $n$ -by $-p$ matrix. Then $A C$ is defined to be the $m$ -by-p matrix whose entry in row $j,$ column $k,$ is given by the following equation:
$$
(A C)_{j, k}=\sum_{r=1}^{n} A_{j, r} C_{r, k}
$$
In other words, the entry in row $j,$ column $k,$ of $A C$ is computed by taking row $j$ of $A$ and column $k$ of $C,$ multiplying together corresponding entries, and then summing.

\defi{$ A_{j, .}, A_{.,k}$} 
Suppose $A$ is an $m$ -by $-n$ matrix.
$\cdot$ If $1 \leq j \leq m,$ then $A_{j,.} \text { denotes the } 1-\text { by }-n \text { matrix consisting of }$ row $j$ of $A$
$\cdot$ If $1 \leq k \leq n,$ then $A., k$ denotes the $m-$ by $-1$ matrix consisting of column $k$ of $A$

\thrm{Entry of matrix product equals row times column}
Suppose $A$ is an $m$-by-$n$ matrix and $C$ is an $n$ -by- $p$ matrix. Then
$$
(A C)_{j, k}=A_{j,.} C_{., k}
$$
for $1 \leq j \leq m$ and $1 \leq k \leq p$

\thrm{Column of matrix product equals matrix times column}
Suppose $A$ is an $m-$ by-n matrix and $C$ is an $n$ -by-p matrix. Then
$$
(A C)_{., k} = A C_{., k}
$$
for $1 \leq k \leq p$

\thrm{Linear combination of columns}
Suppose $A$ is an $m-b y-n$ matrix and $c=\left(\begin{array}{c}{c_{1}} \\ {\vdots} \\ {c_{n}}\end{array}\right)$ is an $n$ -by-1 matrix. Then
$$
A c=c_{1} A_{.1}+\cdots+c_{n} A_{. n}
$$
In other words, $A c$ is a linear combination of the columns of $A,$ with the scalars that multiply the columns coming from $c .$


	\end{multicols*}
\end{document}